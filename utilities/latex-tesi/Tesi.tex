\documentclass[10pt, oneside]{book}

\usepackage[a4paper, top=2.5cm, bottom=2.5cm, left=3cm, right=3cm]{geometry}
\usepackage[utf8]{inputenc}
\usepackage[italian]{babel}
\usepackage{cite}
\usepackage[numbers, sort&compress]{natbib}
\usepackage{hyperref}
\usepackage{graphicx}
\usepackage{fancyhdr}
\usepackage[Lenny]{fncychap}
\usepackage{frontespizio}
\usepackage{tabularx}
\usepackage{caption}
\usepackage{enumitem}
\usepackage{amsmath}

\pagestyle{fancy}
\fancyhead{}
\fancyhead[LE]{\nouppercase{\leftmark}}
\fancyhead[RO]{\nouppercase{\rightmark}}
\bibliographystyle{unsrtnat} %Ordine dei riferimenti bibliografici = ordine con cui le citazioni compaiono nel testo.
\graphicspath{{images/}} %Directory delle immagini

\begin{document}
\begin{frontespizio}
\Universita{Roma Tor Vergata}
\Dipartimento{Ingegneria Civile e Ingegneria dell'Informazione}
\Corso{Ingegneria Informatica}
\Annoaccademico{2023/2024}
\Titolo{Titolo della tesi}
\Candidato[0316179]{Matteo Fanfarillo}
\Relatore{Giuseppe Bianchi}
\Correlatore{Francesco Gringoli}
\Logo{logo.png}
\end{frontespizio}

\begin{flushright}
\null\vspace{\stretch{1}}
\textit{[Citazione]}
\vspace{\stretch{2}}\null
\end{flushright}

\tableofcontents
\listoffigures
\listoftables

\chapter{Introduzione}
\section{Panoramica sull'eSIM}
L'eSIM (embedded-SIM) non è altro che una SIM virtuale: grazie a lei, quando l'utente vuole cambiare operatore, non deve più acquistare fisicamente una nuova SIM card presso un negozio del nuovo operatore, bensì gli è sufficiente ricevere via e-mail un profilo, ossia una "SIM digitale" che può essere caricata subito sul telefono mediante la scansione di un QR code. Si tratta di una soluzione molto più pratica rispetto a recarsi fisicamente presso il negozio dell'operatore, tant'è vero che negli ultimi anni si sta diffondendo sempre di più: uno studio di Juniper Research del 2023 stima che il numero di telefoni che utilizzano la connettività eSIM aumenterà dai 986 milioni del 2023 ai 3.5 miliardi entro il 2027 \cite{Corcom}. Per questi motivi, e poiché le informazioni associate alla comunicazione tra eSIM sono sensibili, è fondamentale garantire un livello di sicurezza sufficientemente elevato per il funzionamento dell'eSIM sia a run-time che a boot-time.

\section{Obiettivo del lavoro}
La presente trattazione si propone di:
\begin{itemize}
\item introdurre una panoramica sul mondo dell'e-SIM, descrivendo tutti i dettagli architetturali e protocollari necessari;
\item effettuare un'analisi di sicurezza e delle vulnerabilità dell'eSIM e del suo funzionamento;
\item se possibile, tentare di sfruttare, anche con delle attività di laboratorio, le eventuali vulnerabilità trovate.
\end{itemize}

\section{Definizioni preliminari}
\begin{itemize}
\item \textbf{eUICC (embedded Universal Integrated Circuit Card)}: è un chip utilizzato nei telefoni all'interno del quale è embeddato il software dell'eSIM. È integrato direttamente nei dispositivi (i.e. non è rimovibile) ed è progettato per essere programmato a distanza. Può contenere uno o più profili eSIM.
\item \textbf{LPA (Local Profile Assistant)}: è un'applicazione che vive nel telefono dell'utente ed è responsabile della gestione dei profili all'interno della rete mobile, inclusi la creazione, l'aggiornamento e la cancellazione.
\item \textbf{SM-DP+ (Subscription Manager Data Preparation plus)}: è un protocollo che rappresenta una tecnica di provisioning usata per configurare le eSIM in modo automatico e remoto. Rispetto alla versione base SM-DP, offre delle funzionalità aggiuntive come un sistema di crittografia più avanzato e un'architettura di rete più flessibile.
\end{itemize}

\section{Panoramica sui capitoli successivi}
Nel capitolo 2 verrà svolta una trattazione dettagliata sull'architettura dell'eSIM e sul protocollo di comunicazione tra eUICC, LPA e server SM-DP+, con lo scopo di fornire al lettore gli strumenti per comprendere appieno le tematiche centrali del lavoro. Nel capitolo 3 verrà effettuata un'analisi della sicurezza dell'eSIM a boot-time (i.e. durante la fase di configurazione). Nel capitolo 4 verranno mostrati i risultati ottenuti. Infine, nel capitolo 5 verranno tratte delle conclusioni sul lavoro svolto e verrà fornita una panoramica sui possibili progetti futuri che potranno essere intrapresi a partire dai risultati ottenuti attraverso questo lavoro.

\chapter{Interfacce e funzionamento dell'eSIM}
\section{Architettura di RSP}
Per comprendere appieno come funziona e come si interfaccia l'eSIM all'interno dei dispositivi mobili, è necessario introdurre il protocollo \textbf{RSP}, anche perché l'eSIM si colloca proprio all'interno dell'architettura di RSP.\\
RSP (Remote SIM Provisioning) è un protocollo utilizzato dal protocollo SM-DP+ per gestire la comunicazione tra il server SM-DP+ e la scheda eSIM del dispositivo mobile (i.e. l'eUICC). In particolare, definisce le operazioni di provisioning specifiche per la comunicazione dell'eUICC. Quest'ultimo comprende i dati sia dell'operatore che dell'utente che, nel caso delle SIM tradizionali, verrebbero memorizzati su una SIM card fisica. L'end user che vuole ottenere un profilo eSIM offerto da un particolare operatore (nel quale viene definito un piano tariffario) deve pagare l'operatore affinché esso gli fornisca un codice QR. Dopodiché, deve effettuare la scansione del codice QR per avviare lo scaricamento (operazione di Download) e l'installazione (operazione di Install) del profilo eSIM: a questo punto, la connessione tra end user (col relativo profilo eSIM) e operatore è completata. Se in un secondo momento l'end user ha la necessità di ottenere un secondo profilo eSIM, gli è sufficiente ripetere i medesimi passaggi appena descritti, e questo secondo profilo può essere installato all'interno del medesimo eUICC che ospita già il primo profilo. Tale meccanismo è illustrato nella figura \ref{fig:RSP-functioning} tratta da \cite{GSMA-whitepaper}.
\begin{figure}
\includegraphics[width=\linewidth]{RSP-functioning.png}
\caption{Comunicazione tra end user e operatore nel contesto del Remote SIM Provisioning.}
\label{fig:RSP-functioning}
\end{figure}
\\Per quanto riguarda l'architettura interna di RSP nello specifico, esistono due soluzioni diverse \cite{GSMA-docs-new}.
\begin{enumerate}
\item \textbf{LPA embeddato nel dispositivo mobile ma non all'interno dell'eUICC (LPAd)}: oltre alla comunicazione tra l'applicazione LPA e SM-DP+, si utilizzano delle apposite interfacce anche per la comunicazione tra l'eUICC e l'applicazione LPA, come mostrato nella figura \ref{fig:RSP-LPAd} tratta da \cite{GSMA-docs-new}.
\begin{figure}
\includegraphics[width=\linewidth]{RSP-LPAd.png}
\caption{Architettura di RSP nel caso di LPA non embeddato nell'eUICC.}
\label{fig:RSP-LPAd}
\end{figure}
Di seguito è riportato un breve glossario che chiarisce il significato di alcuni componenti appartenenti all'architettura di RSP raffigurata in \ref{fig:RSP-LPAd}.
\begin{itemize}
\item \textbf{CI} = Certificate Issuer: nota anche come eSIM CA RootCA, è un'entità autorizzata a rilasciare certificati digitali.
\item \textbf{Device App} = una qualunque applicazione installata nel dispositivo mobile.
\item \textbf{Enterprise} = impresa (i.e. azienda, organizzazione o entità governativa) che si iscrive ai servizi mobili che devono essere utilizzati dai dipendenti a supporto dell'impresa stessa.
\item \textbf{EUM} = eUICC Manufacturer: è il fornitore delle eUICC e del software residente (e.g. firmware, sistema operativo); svolge anche il ruolo di certificate authority subordinata al CI e rilascia certificati all'eUICC \cite{GSMA-docs-new}\cite{Sec-analysis}.
\item \textbf{HRI Server} = server che fornisce le High Resolution Icon, che sono icone che vengono create per essere visualizzate in alta risoluzione.
\item \textbf{LDSd} = Local Discovery Service (quando LPA non è nell'eUICC).
\item \textbf{LPDd} = Local Profile Download (quando LPA non è nell'eUICC).
\item \textbf{LUId} = Local User Interface (quando LPA non è nell'eUICC).
\item \textbf{SM-DS} = Subscription Manager Discovery Server: è il componente che consente a SM-DP+ di raggiungere l'eUICC senza dover sapere a quale rete il dispositivo è connesso.
\end{itemize}
\item \textbf{LPA embeddato all'interno dell'eUICC (LPAe)}: sono necessarie solo delle interfacce tra l'eUICC e SM-DP+, come mostrato nella figura \ref{fig:RSP-LPAe} tratta da \cite{GSMA-docs-new}.
\begin{figure}
\includegraphics[width=\linewidth]{RSP-LPAe.png}
\caption{Architettura di RSP nel caso di LPA embeddato nell'eUICC.}
\label{fig:RSP-LPAe}
\end{figure}
Successivamente è riportato un breve glossario che chiarisce il significato di alcuni componenti appartenenti all'architettura di RSP raffigurata in \ref{fig:RSP-LPAe}.
\begin{itemize}
\item \textbf{LDSe} = Local Discovery Service (quando LPA è nell'eUICC).
\item \textbf{LPDe} = Local Profile Download (quando LPA è nell'eUICC).
\item \textbf{LUIe} = Local User Interface (quando LPA è nell'eUICC).
\item \textbf{UIMe} = User Inteface Module.
\end{itemize}
\end{enumerate}

\subsection{Interfacce presenti nell'architettura di RSP}
Sono illustrate nella tabella \ref{tab:interfaces}, costruita a partire da informazioni tratte da \cite{GSMA-docs-new}.\\
\begin{table}[h!]
\begin{center}
\captionsetup{skip=4pt}
\caption{Interfacce in RSP}
\label{tab:interfaces}
\begin{tabularx}{\textwidth}{|c|c|c|X|} % <-- Alignments: 1st column center, 2nd center and 3rd center, with vertical lines in between
\hline
\textbf{Interfaccia} & \textbf{Componente 1} & \textbf{Componente 2} & \textbf{Descrizione}\\
\hline
ES2+ & Operatore & SM-DP+ & Viene usata dall'operatore per invocare la preparazione del Profile Package*.\\
\hline
ES6 & Operatore & eUICC & Viene usata dall'operatore per gestire il contenuto dei profili.\\
\hline
ES8+ & SM-DP+ & eUICC & Fornisce un canale end-to-end sicuro tra SM-DP+ e l'eUICC per l'amministrazione dell'ISD-P** e del relativo profilo durante il download e l'installazione.\\
\hline
ES9+ & SM-DP+ & LPD & Viene usata per fornire trasporto sicuro tra SM-DP+ e LPD per la consegna del Profile Package.\\
\hline
ES10a & LDSd & eUICC & Viene usata da LPAd per ottenere gli indirizzi configurati dall'eUICC per Root SM-DS*** (gestione di una Discovery Request).\\
\hline
ES10b & LPDd & eUICC & Viene usata da LPAd per trasferire un Profile Package all'eUICC.\\
\hline
ES10c & LUId & eUICC & Viene usata da LPAd per la gestione locale dei profili installati sull'eUICC da parte dell'end user (e.g. Enable, Disable, Delete).\\
\hline
ES11 & LDS & SM-DS & Viene usata per l'ottenimento di eventi.\\
\hline
ES12 & SM-DP+ & SM-DS & Viene usata per la gestione degli eventi.\\
\hline
ES15 & SM-DS & SM-DS & Viene usata per connettere gli SM-DS tra loro nel caso in cui ce ne sia più di uno.\\
\hline
ES22 & LPAd & Device App & Viene usata da un'applicazione del dispositivo mobile per interoperare con l'LPA.\\
\hline
ES25 & UIMe & LUIe & Viene usata per trasferire verso l'LPA le interazioni dell'end user.\\
\hline
ESop & Operatore & End user & È specifica per le relazioni di business tra l'operatore e l'end user.\\
\hline
ESeu & End user & LUI & È specifica per le relazioni di business tra l'end user e la LUI.\\
\hline
ESeum & eUICC & EUM & È specifica per le relazioni di business tra l'eUICC e l'EUM.\\
\hline
ESci & CI & SM-DP+, SM-DS, EUM & Viene usata per richiedere certificati.\\
\hline
EShri & LUI & HRI Server & Viene usata per recuperare le High Resolution Icon.\\
\hline
ESent & Operatore & Enterprise & È un'interfaccia che prescinde dagli scopi del presente documento.\\
\hline
ESapp & Operatore & Device App & È un'interfaccia che prescinde dagli scopi del presente documento.\\
\hline
\end{tabularx}
\end{center}
\end{table}
\\\textit{*Un Profile Package è un pacchetto di dati associato a un profilo che contiene le informazioni di configurazione necessarie per attivare e utilizzare quel profilo all'interno di una scheda eSIM. Esistono diversi tipi di Profile Package: l'Unprotected Profile Package (UPP) è un pacchetto di dati non protetto da alcun meccanismo di sicurezza; il Protected Profile Package (PPP) è un pacchetto di dati protetto da alcuni meccanismi di sicurezza, come l'autenticazione e/o la crittografia; il Bound Profile Package (BPP) è un pacchetto di dati legato a un particolare dispositivo o a una piattaforma di servizi; il Segmented Bound Profile Package (SBPP), infine, non è altro che un BPP suddiviso in molteplici segmenti che possono essere utilizzati in modo indipendente e separato.}\\ \\
\textit{**ISD-P (Issuer Security Domain Profile) è un contenitore sicuro che ospita un unico profilo.}\\ \\
\textit{***Root SM-DS è il server primario utilizzato da un operatore di rete mobile per gestire le attivazioni e le disattivazioni delle sottoscrizioni e per gestire funzionalità come l'autenticazione e l'autorizzazione degli utenti. D'altra parte, si hanno gli Alternative SM-DS, che sono server di backup a cui si ricorre quando il Root SM-DS non è disponibile.}

\section{Architettura dell'eUICC}
Nella figura \ref{fig:eUICC-arch} tratta da \cite{GSMA-docs-new} è schematizzata l'architettura interna del chip eUICC, dove i riquadri e le frecce in rosso sono relativi rispettivamente ai componenti e alle interfacce che, nell'ambito dell'eUICC, sono presenti esclusivamente nel caso in cui l'applicazione LPA sia effettivamente embeddata all'interno dell'eUICC (LPAe).
\begin{figure}
\includegraphics[width=\linewidth]{eUICC-arch.png}
\caption{Architettura dell'eUICC.}
\label{fig:eUICC-arch}
\end{figure}
\\Di seguito, invece, è riportato un breve glossario che chiarisce il significato di alcuni componenti appartenenti all'architettura dell'eUICC raffigurata in \ref{fig:eUICC-arch}.
\begin{itemize}
\item \textbf{CASD} = Controller Authority Security Domain: è un'area di storage sicura all'interno dell'ISD-P in cui vengono memorizzate le credenziali richieste (i.e. chiavi, certificati) per supportare le funzionalità di sicurezza sensibili.
\item \textbf{ECASD} = Embedded Controller Authority Security Domain: è il componente CASD direttamente incapsulato all'interno dell'eUICC.
\item \textbf{ISD-R} = Issuer Security Domain Root: è il componente responsabile della creazione di nuovi ISD-P e della gestione del loro ciclo di vita.
\item \textbf{LPA Services} = i seguenti quattro servizi: trasferimento del Bound Profile Package da LPAd all'ISD-P; ottenimento della lista dei profili installati; recupero dell'EID (eUICC ID); ottenimento delle operazioni di gestione del profilo locale (Local Profile Management Operations).
\item \textbf{MNO-SD} = Mobile Network Operator Security Domain: è la parte del profilo posseduta dall'operatore che fornisce all'operatore Over The Air (OTA) un canale di comunicazione sicuro; viene usato per gestire il contenuto di un profilo una volta che è stato abilitato.
\item \textbf{NAAs} = Network Access Applications: sono le applicazioni che consentono l'accesso alla rete.
\item \textbf{Profile Package Interpreter} = servizio del sistema operativo dell'eUICC che traduce i dati del Profile Package in un profilo installato all'interno dell'ISD-P usando il formato interno dell'eUICC.
\item \textbf{Profile Policy Enabler} = componente che verifica che il profilo eSIM possa essere installato sull’eUICC.
\item \textbf{SSD} = Supplementary Security Domain: è un'area di memoria protetta all'interno dell'ISD-P che viene utilizzata per l'esecuzione di funzioni di sicurezza come le operazioni crittografiche. Di fatto, il suo scopo principale è quello di proteggere le informazioni riservate dell'utente (i.e. chiavi, password) da accessi non autorizzati e attacchi esterni.
\item \textbf{Telecom Framework} = servizio del sistema operativo dell'eUICC che fornisce algoritmi di autenticazione di rete standardizzati alle applicazioni NAAs ospitate nei rispettivi ISD-P.
\end{itemize}

\subsection{Caratteristiche hardware e software dell'eUICC}
\begin{enumerate}[itemsep=0pt]
\item Deve essere resistente al tampering dei componenti hardware.
\item Contiene un unico ECASD (eUICC Controlling Authority Security Domain).
\item Supporta SHA-1.
%\item Supporta TUAK, che è un particolare algoritmo crittografico di 3GPP, dove 3GPP (Third Generation Partnership Project) è il consorzio industriale che definisce gli standard per la tecnologia 5G .
\item Supporta Milenage, che è un set di funzioni di autenticazione e di generazione di chiavi.
\item Tutte le funzioni crittografiche devono essere resistenti al tampering e agli attacchi side-channel.
\end{enumerate}

\section{Chiavi crittografiche e certificati}
\subsection{Chiavi crittografiche}\label{sec:keys}
I principali attori che interagiscono nel protocollo RSP sono l'eUICC, l'LPA e il server SM-DP+. Le chiavi utilizzate da loro hanno tutte un nome di tipo $<$XX$>$.$<$YY$>$.$<$ZZ$>$ \cite{GSMA-docs-new}, dove:
\begin{itemize}
\item \textbf{$<$XX$>$}: indica la natura della chiave (i.e. chiave pubblica PK, chiave privata SK, chiave pubblica one-time otPK, chiave privata one-time otSK).
\item \textbf{$<$YY$>$}: indica il proprietario della chiave.
\item \textbf{$<$ZZ$>$}: indica l'utilizzo della chiave (i.e. digital signature SIG, key agreement KA, TLS).
\end{itemize}
Le chiavi di maggiore rilievo, comunque sia, sono riportate nella tabella \ref{tab:keys} tratta da \cite{GSMA-docs-old}.
\begin{table}[h!]
\begin{center}
\captionsetup{skip=4pt}
\caption{Chiavi crittografiche in RSP}
\label{tab:keys}
\begin{tabularx}{\textwidth}{|c|X|}
\hline
\textbf{Nome} & \textbf{Descrizione}\\
\hline
PK.EUICC.SIG & Chiave pubblica dell'eUICC usata per verificare le signature dell'eUICC. È inclusa nel certificato CERT.EUICC.SIG.\\
\hline
SK.EUICC.SIG & Chiave privata dell'eUICC usata per generare le signature.\\
\hline
PK.DPauth.SIG & Chiave pubblica del server SM-DP+ usata per verificare le signature del server in fase di autenticazione. È inclusa nel certificato CERT.DPauth.SIG.\\
\hline
SK.DPauth.SIG & Chiave privata del server SM-DP+ usata per generare le signature per autenticarsi all'eUICC.\\
\hline
PK.DPpb.SIG & Chiave pubblica del server SM-DP+ usata per verificare le signature del server comprese nel BPP.  È inclusa nel certificato CERT.DPpb.SIG.\\
\hline
SK.DPpb.SIG & Chiave privata del server SM-DP+ usata per generare le signature per il binding dei profili.\\
\hline
PK.DSauth.SIG & Chiave pubblica del server SM-DS usata per verificare le signature di SM-DS in fase di autenticazione. È inclusa nel certificato CERT.DSauth.SIG.\\
\hline
SK.DSauth.SIG & Chiave privata del server SM-DS usata per generare le signature per autenticarsi all'eUICC.\\
\hline
PK.EUM.SIG & Chiave pubblica dell'EUM usata per verificare i certificati degli eUICC. È inclusa nel certificato CERT.EUM.SIG.\\
\hline
SK.EUM.SIG & Chiave privata dell'EUM usata per firmare i certificati degli eUICC.\\
\hline
PK.CI.SIG &  Chiave pubblica del CI usata per verificare i certificati dell'EUM, dei server SM-DS e del server SM-DP+.\\
\hline
SK.CI.SIG & Chiave privata del CI usata per firmare i certificati dell'EUM, dei server SM-DS e del server SM-DP+.\\
\hline
otPK.EUICC.KA & Chiave pubblica one-time dell'eUICC usata per il key agreement.\\
\hline
otSK.EUICC.KA & Chiave privata one-time dell'eUICC usata per il key agreement.\\
\hline
otPK.DP.KA & Chiave pubblica one-time del server SM-DP+ usata per il key agreement.\\
\hline
otSK.DP.KA & Chiave privata one-time del server SM-DP+ usata per il key agreement.\\
\hline
PK.DP.TLS & Chiave pubblica del server SM-DP+ usata per verificare le signature TLS del server. È inclusa nel certificato CERT.DP.TLS.\\
\hline
SK.DP.TLS & Chiave privata del server SM-DP+ usata per generare le signature TLS per autenticarsi all'LPA.\\
\hline
PK.DS.TLS & Chiave pubblica del server SM-DS usata per verificare le signature TLS di SM-DS. È inclusa nel certificato CERT.DS.TLS.\\
\hline
SK.DS.TLS & Chiave privata del server SM-DS usata per generare le signature TLS per autenticarsi all'LPA.\\
\hline
\end{tabularx}
\end{center}
\end{table}

\subsection{Certificati}
I certificati propri dei principali componenti che partecipano all'interazione data dal protocollo RSP sono riportati nella tabella \ref{tab:cert} tratta da \cite{GSMA-docs-new}.\\
\begin{table}[h!]
\begin{center}
\captionsetup{skip=4pt}
\caption{Certificati in RSP}
\label{tab:cert}
\begin{tabularx}{\textwidth}{|c|X|X|}
\hline
\textbf{Nome} & \textbf{Descrizione} & \textbf{Note}\\
\hline
CERT.CI.SIG & Certificato GSMA CI & Viene firmato e rilasciato da se stesso.\\
\hline
CERT.CISubCA.SIG & Certificato GSMA CI subordinato & Se esiste, viene firmato e rilasciato dal CI root.\\
\hline
CERT.EUM.SIG & Certificato EUM & Viene firmato e rilasciato dal CI (root o subordinato).\\
\hline
CERT.EUMSubCA.SIG & Certificato EUM subordinato & Se esiste, viene firmato e rilasciato dall'EUM root.\\
\hline
CERT.DPSubCA.SIG & Certificato SM-DP+ intermediario & Viene firmato e rilasciato dal CI (root o subordinato).\\
\hline
CERT.DPauth.SIG & Certificato SM-DP+ per autenticarsi all'eUICC & Viene firmato e rilasciato dal SM-DP+ intermediario o dal CI (root o subordinato).\\
\hline
CERT.DPpb.SIG & Certificato SM-DP+ per rilasciare e firmare i profili eSIM & Viene firmato e rilasciato dal SM-DP+ intermediario o dal CI (root o subordinato).\\
\hline
CERT.DP.TLS & Certificato TLS di SM-DP+ & Viene firmato e rilasciato dal SM-DP+ intermediario o dal CI (root o subordinato).\\
\hline
CERT.DSSubCA.SIG & Certificato SM-DS intermediario & Viene firmato e rilasciato dal CI (root o subordinato).\\
\hline
CERT.DSauth.SIG & Certificato SM-DS & Viene firmato e rilasciato dal SM-DS intermediario o dal CI (root o subordinato).\\
\hline
CERT.DS.TLS & Certificato TLS di SM-DS & Viene firmato e rilasciato dal SM-DS intermediario o dal CI (root o subordinato).\\
\hline
CERT.EUICC.SIG & Certificato eUICC & Viene firmato e rilasciato dall'EUM (root o subordinato).\\
\hline
CERT.CA.SIG & Certificato di una qualunque CA pubblica & Può firmare e rilasciare certificati TLS.\\
\hline
\end{tabularx}
\end{center}
\end{table}
\\La figura \ref{fig:cert-chain} tratta da \cite{GSMA-docs-old} definisce uno schema riassuntivo della struttura originale della catena di certificati definita dalla Public Key Infrastructure (PKI) di RSP.\\
Attualmente, invece, la struttura della certificate chain è più complessa e introduce molteplici varianti differenti (i.e. presenza o meno di GSMA CI subordinati, presenza o meno di EUM subordinati, presenza o meno di SM-DP+ intermediari, presenza o meno di SM-DS intermediari). Di seguito, per semplicità, più porzioni distinte della catena verranno analizzate separatamente.
\begin{itemize}
\item Porzione della catena comprendente il CI, l'EUM e l'eUICC: è riportata nella figura \ref{fig:cert-chain-new1} tratta da \cite{GSMA-docs-new} e prevede quattro varianti differenti.
\begin{itemize}
\item \textbf{Variante O (originale)}: il CI root rilascia certificati per l'EUM root, il quale rilascia a sua volta certificati per l'eUICC.
\item \textbf{Variante A}: il CI root rilascia certificati per l'EUM root; l'EUM root rilascia certificati per l'EUM subordinato; l'EUM subordinato, infine, rilascia certificati per l'eUICC.
\item \textbf{Variante B}: il CI root rilascia certificati per il CI subordinato; il CI subordinato rilascia certificati per l'EUM root; l'EUM root, infine, rilascia certificati per l'eUICC.
\item \textbf{Variante C}: il CI root rilascia certificati per il CI subordinato; il CI subordinato rilascia certificati per l'EUM root; l'EUM root rilascia certificati per l'EUM subordinato; l'EUM subordinato, infine, rilascia certificati per l'eUICC.
\end{itemize}
\item Porzione della catena comprendente il CI e i certificati di tipo SIG di SM-DP+ e SM-DS: è riportata nella figura \ref{fig:cert-chain-new2} tratta da \cite{GSMA-docs-new} e prevede quattro varianti differenti.
\begin{itemize}
\item \textbf{Variante O (originale)}: il CI root rilascia direttamente i certificati CERT.DPauth.SIG, CERT.DPpb.SIG, CERT.DSauth.SIG.
\item \textbf{Variante A}: il CI root rilascia certificati per l'SM-DP+ intermediario e l'SM-DS intermediario; l'SM-DP+ intermediario rilascia i certificati  CERT.DPauth.SIG, CERT.DPpb.SIG; l'SM-DS intermediario rilascia il certificato CERT.DSauth.SIG.
\item \textbf{Variante B}: il CI root rilascia certificati per il CI subordinato, il quale rilascia a sua volta i certificati CERT.DPauth.SIG, CERT.DPpb.SIG, CERT.DSauth.SIG.
\item \textbf{Variante C}: il CI root rilascia certificati per il CI subordinato; il CI subordinato rilascia certificati per l'SM-DP+ intermediario e l'SM-DS intermediario; l'SM-DP+ intermediario rilascia i certificati CERT.DPauth.SIG, CERT.DPpb.SIG; l'SM-DS intermediario rilascia il certificato CERT.DSauth.SIG.
\end{itemize}
\item Porzione della catena comprendente il CI e i certificati TLS di SM-DP+ e SM-DS: è riportata nella figura \ref{fig:cert-chain-new3} tratta da \cite{GSMA-docs-new} e prevede quattro varianti differenti.
\begin{itemize}
\item \textbf{Variante O (originale)}: il CI root rilascia direttamente i certificati CERT.DP.TLS, CERT.DS.TLS.
\item \textbf{Variante A}: il CI root rilascia i certificati per l'SM-DP+ intermediario e l'SM-DS intermediario; l'SM-DP+ intermediario rilascia il certificato CERT.DP.TLS; l'SM-DS intermediario rilascia il certificato CERT.DS.TLS.
\item \textbf{Variante B}: il CI root rilascia certificati per il CI subordinato, il quale rilascia a sua volta i certificati CERT.DP.TLS, CERT.DS.TLS.
\item \textbf{Variante C}: il CI root rilascia certificati per il CI subordinato; il CI subordinato rilascia certificati per l'SM-DP+ intermediario e l'SM-DS intermediario; l'SM-DP+ intermediario rilascia il certificato CERT.DP.TLS; l'SM-DS intermediario rilascia il certificato CERT.DS.TLS.
\end{itemize}
\end{itemize}
Per quanto riguarda i certificati TLS, al posto del CI, può esserci come root qualunque CA pubblica, come mostrato nella figura \ref{fig:cert-chain-new4} tratta da \cite{GSMA-docs-new}. Anche con questa soluzione esistono diverse varianti.
\begin{itemize}
\item \textbf{Prima variante OO (originale)}: la CA pubblica root rilascia direttamente i certificati CERT.DP.TLS, CERT.DS.TLS.
\item \textbf{Seconda variante OO}: la CA pubblica root rilascia certificati per la CA pubblica subordinata, la quale rilascia a sua volta i certificati CERT.DP.TLS, CERT.DS.TLS.
\item \textbf{Prima variante AA}: la CA pubblica rilascia certificati per l'SM-DP+ intermediario e l'SM-DS intermediario; l'SM-DP+ intermediario rilascia il certificato CERT.DP.TLS; l'SM-DS intermediario rilascia il certificato CERT.DS.TLS.
\item \textbf{Seconda variante AA}: la CA pubblica rilascia certificati per la CA pubblica subordinata; la CA pubblica subordinata rilascia certificati per l'SM-DP+ intermediario e l'SM-DS intermediario; l'SM-DP+ intermediario rilascia il certificato CERT.DP.TLS; l'SM-DS intermediario rilascia il certificato CERT.DS.TLS.
\end{itemize}
\begin{figure}
\includegraphics[width=\linewidth]{cert-chain.png}
\caption{Catena di certificati definita originariamente dalla PKI di RSP.}
\label{fig:cert-chain}
\end{figure}
\begin{figure}
\includegraphics[width=\linewidth]{cert-chain-new1.png}
\caption{Prima porzione dell'attuale catena di certificati definita dalla PKI di RSP.}
\label{fig:cert-chain-new1}
\end{figure}
\begin{figure}
\includegraphics[width=\linewidth]{cert-chain-new2.png}
\caption{Seconda porzione dell'attuale catena di certificati definita dalla PKI di RSP.}
\label{fig:cert-chain-new2}
\end{figure}
\begin{figure}
\includegraphics[width=\linewidth]{cert-chain-new3.png}
\caption{Terza porzione dell'attuale catena di certificati definita dalla PKI di RSP.}
\label{fig:cert-chain-new3}
\end{figure}
\begin{figure}
\includegraphics[width=\linewidth]{cert-chain-new4.png}
\caption{Quarta porzione dell'attuale catena di certificati definita dalla PKI di RSP.}
\label{fig:cert-chain-new4}
\end{figure}
All'interno della PKI, il Certificate Issuer di GSMA (CI) è la Root Certification Authority del servizio RSP e, di conseguenza, rappresenta il nodo radice della catena. Inoltre, tutti i certificati possono essere rilasciati direttamente dal CI (root o subordinato), fatta eccezione di CERT.EUICC.SIG che, invece, viene rilasciato dall'EUM (root o subordinato). Tutti i certificati che possono essere rilasciati direttamente dal CI hanno la possibilità di essere revocati in qualunque momento, in particolar modo se le entità corrispondenti (CI, EUM, SM-DP+, SM-DS) vengono compromesse. D'altra parte, i certificati eUICC (CERT.EUICC.SIG) non vengono revocati in modo individuale: di fatto, è difficile che un singolo eUICC venga compromesso. Piuttosto, è più verosimile che un modello eUICC o un intero batch di produzione di eUICC venga dichiarato come compromesso; quando ciò avviene, quello che si fa è revocare direttamente il certificato EUM (CERT.EUM.SIG) associato a quel modello o batch di produzione di eUICC \cite{GSMA-docs-new}.\\
Il CI fornisce una Certificate Revocation List (CRL), che è la lista dei certificati revocati tra tutti i certificati non scaduti che erano stati rilasciati da quello stesso CI. Ciascun CI, per giunta, deve pubblicare la propria CRL sia periodicamente, sia ogni volta che viene revocato un particolare certificato \cite{GSMA-docs-new}.\\
In realtà, i certificati relativi alla PKI di RSP di cui si è discusso finora non sono gli unici certificati utilizzati per effettuare il deployment dei profili eSIM: esiste anche un certificato per firmare l'applicazione LPA e un certificato da inserire in ciascun profilo eSIM da distribuire all'end user. Dove entrano in gioco tali certificati? Con riferimento alla guida di Android per le API e per l'implementazione del deployment dei profili eSIM \cite{Android-docs}, l'interazione tra un'applicazione LPA e l'interfaccia all'eUICC (legata al componente EuiccManager in \cite{Android-docs}) può avvenire solo se l'applicazione dispone dei privilegi dell'operatore. Di norma, tali privilegi sono conferiti all'applicazione se il certificato usato per firmarla coincide col certificato presente nel profilo fornito da SM-DP+.

\subsection{Aggiornamento delle chiavi pubbliche nell'eUICC}
L'eUICC può fornire un meccanismo per aggiornare il set di chiavi pubbliche memorizzate nell'ECASD dell'eUICC (che sono chiavi a lunga durata). L'implementazione di tale meccanismo è lasciata all'EUM o al produttore del dispositivo mobile e, stando alla guida ufficiale di GSMA \cite{GSMA-docs-new}, deve essere sicura. Tuttavia, nel momento in cui un'implementazione è affidata a terze parti, nessuno può garantire con certezza che la prescrizione sulla sicurezza venga rispettata, poiché non vengono seguite delle linee guida standard e consolidate. Di conseguenza, nei capitoli successivi potrebbe essere utile approfondire mediante degli esperimenti pratici il funzionamento dell'aggiornamento del set di chiavi poste nell'ECASD e stabilire così se esiste qualche vulnerabilità da sfruttare a vantaggio di un attaccante.

\subsubsection{Contenuto dell'ECASD}
Tutti gli eUICC devono avere un ECASD che contenga \cite{GSMA-docs-new}:
\begin{itemize}[itemsep=0pt]
\item la chiave privata dell'eUICC (SK.EUICC.SIG);
\item il certificato dell'eUICC (CERT.EUICC.SIG) contenente la chiave PK.EUICC.SIG, che è la chiave pubblica dell'eUICC;
\item la chiave pubblica del CI root (PK.CI.SIG);
\item il certificato dell'EUM (CERT.EUM.SIG) e, opzionalmente, il certificato degli EUM subordinati (CERT.EUMSubCA.SIG);
\item un keyset dell'EUM per il rinnovo di chiavi e certificati.
\end{itemize}

\subsubsection{Quando avviene l'aggiornamento delle chiavi?}
La necessità di rinnovare le chiavi contenute nell'ECASD può presentarsi in due casi \cite{GSMA-docs-new}.
\begin{itemize}
\item L'applicazione LPA ha determinato che il Public Key identificator (i.e. la rappresentazione in esadecimale dell'identificatore della chiave pubblica) del CI non è supportato dall'eUICC.
\item Durante la procedura di Common Mutual Authentication, il server SM-DP+ ha restituito un certificato CERT.DPauth.SIG che ha come parent un Root Certificate non supportato dall'eUICC.
\end{itemize}

\section{Interazione tra eUICC, LPA, SM-DP+ e operatore}
\subsection{Sicurezza TLS}
Il protocollo TLS, la cui versione 1.2 è definita in RFC 5246 \cite{RFC-5246} e la cui versione 1.3 è definita in RFC 8446 \cite{RFC-8446}, è utilizzato per proteggere il traffico sulle interfacce ES2+ (tra server SM-DP+ e operatore) e ES9+ (tra server SM-DP+ e LPA), dove è prevista la mutua autenticazione tra le parti. La documentazione di GSMA di riferimento \cite{GSMA-docs-new} sottolinea l'obbligatorietà di fare uso di TLS v1.2 sia per gli algoritmi di autenticazione e autorizzazione, sia per l'integrità dei messaggi, sia per la confidenzialità. In realtà, introduce anche la possibilità (e suggerisce) di utilizzare TLS v1.3, che è la versione più recente di TLS e risolve le vulnerabilità che caratterizzano TLS v1.2, per cui, in linea di principio, dovrebbe risultare particolarmente difficoltoso da penetrare. Tuttavia, attualmente sembra essere solo un suggerimento, per cui nei capitoli successivi potrebbe essere necessario verificare a livello pratico qual è la versione di TLS utilizzata per proteggere la comunicazione tra LPA e server SM-DP+.\\
Un discorso analogo vale per l'interazione che si ha nella mutua autenticazione tra l'eUICC e il server SM-DP+, dove le due parti interagiscono tra loro tramite un TLS tunnel \cite{Sec-analysis}. Per quanto invece riguarda la comunicazione tra eUICC e applicazione LPA, è richiesto l'utilizzo di un pairwise secure channel (ovvero di un canale di comunicazione sicuro rispetto alla confidenzialità e all'autenticazione dei messaggi) che collega le due parti all'interno del dispositivo mobile \cite{Sec-analysis}. Tuttavia, non esiste una specifica universale che imponga l'utilizzo di un particolare protocollo di crittografia per proteggere il pairwise secure channel; il protocollo più utilizzato a tal proposito rimane TLS ma, di nuovo, potrebbe essere necessario stabilire con un approccio pratico se nel canale di comunicazione tra eUICC e LPA viene utilizzato TLS oppure un protocollo differente.

\subsection{Regole per la comunicazione in RSP}
Qualunque comunicazione remota definita per RSP deve far fede alle regole riportate di seguito \cite{GSMA-docs-new}.
\begin{itemize}
\item \textbf{Mutua autenticazione tra eUICC e server SM-DP+}: il server deve essere autenticato per primo da parte dell'eUICC, dove il processo di autenticazione deve includere la verifica di una catena di certificati del server. D'altra parte, l'eUICC deve essere autenticato in un secondo momento da parte del server, dove il processo di autenticazione, di nuovo, deve includere la verifica di una catena di certificati dell'eUICC; l'autenticazione dell'eUICC non si applica all'LPA.
\item \textbf{Privacy dei dati}: l'eUICC, in quanto client, non deve rivelare alcuna informazione privata a un server SM-DP+ non autenticato. Inoltre, non deve generare materiale firmato prima del completamento del processo di autenticazione del server.
\item \textbf{Protezione della comunicazione}: quando possibile, la comunicazione tra eUICC e server SM-DP+, oltre a essere protetta dall'integrità dei messaggi, dalla cifratura e dall'autenticazione del mittente, dovrebbe essere caratterizzata dalla proprietà di Perfect Forward Secrecy. Secondo tale proprietà, se anche una chiave a lungo termine viene compromessa, le chiavi di sessione generate a partire da essa rimangono comunque riservate.
\item \textbf{Autorizzazione}: il server SM-DP+ deve sempre verificare che il client che ha inviato una richiesta sia effettivamente autorizzato prima di far partire l'esecuzione della funzione desiderata.
\end{itemize}

\subsection{Step dell'interazione in RSP}\label{sec:step-RSP}
L'interazione tra le parti, nel contesto del protocollo RSP, avviene in quattro fasi distinte: profile ordering, download initialization, common handshake e profile download \cite{Sec-analysis}.
\begin{itemize}
\item \textbf{Profile ordering \& download initialization}: nella prima fase, l'operatore richiede al server SM-DP+ di preparare un profilo eSIM, e il server gli restituisce dei download initialization pointer (che possono essere rappresentati ad esempio da un codice di attivazione). Nella seconda fase, l'operatore consegna i download initialization pointer all'LPA in modo tale che poi sia possibile effettuare il download vero e proprio del profilo. Per questi primi due step, si possono seguire tre tipi di approcci differenti:
\begin{enumerate}
\item \underline{Default server approach}: la figura \ref{fig:default-server} tratta da \cite{Sec-analysis} mostra questo approccio. L'operatore (indicato con MNO - Mobile Network Operator) pre-condivide con l'eUICC l'indirizzo S del server. Quando vuole effettuare una nuova sottoscrizione e ottenere così un nuovo profilo, l'end user contatta l'operatore (messaggio 0), il quale ordina al server SM-DP+ il profilo per l'identificatore U dell'eUICC target (messaggio 1). Il server crea il nuovo profilo e lo invia all'operatore (messaggio 2), il quale notifica l'utente (messaggio 3). A tal punto, l'applicazione LPA viene avviata e, tramite un'operazione di get, recupera S dall'eUICC.
\item \underline{Activation Code approach}: la figura \ref{fig:activation-code} tratta da \cite{Sec-analysis} mostra questo secondo approccio. L'operatore ordina al server SM-DP+ dei profili (messaggio 1), e il server li rende disponibili con un codice di attivazione (tipicamente un codice QR) e li restituisce all'operatore (messaggio 2). Il codice di attivazione include l'indirizzo S del server, l'identificatore $I_{ac}$ del profilo e, opzionalmente, l'OID, ovvero l'identificatore del server SM-DP+. Quando l'end user vuole effettuare una nuova sottoscrizione e ottenere così un nuovo profilo, contatta l'operatore (messaggio 0, che può essere inviato sia prima che dopo l'interazione tra operatore e SM-DP+ server appena descritta), il quale restituisce il codice di attivazione opportuno (messaggio 3).
\item \underline{SM-DS assisted approach}: è un approccio analogo all'Activation Code, con la differenza che SM-DP+ si appoggia sui server SM-DS per comunicare con l'eUICC.
\end{enumerate}
\item \textbf{Common handshake}: questa fase, nota anche come Common Mutual Authentication, coinvolge tre attori fondamentali: il server SM-DP+, l'applicazione LPA e l'eUICC. I relativi dettagli sono riportati nella sezione \ref{sec:mutual-auth}.
\item \textbf{Profile download}: in quest'ultima fase, i cui dettagli sono riportati nella sezione \ref{sec:down-install}, viene calcolato uno shared secret da cui, mediante una key derivation function (KDF), saranno derivate le chiavi di sessione k (per la cifratura) e k' (per l'integrità dei dati). Avviene poi il download del profilo eSIM, in cui il server SM-DP+ invia all'LPA il profilo cifrato con la chiave k e l'operatore di riferimento, dove entrambe le informazioni sono firmate singolarmente con la chiave k' mediante il meccanismo di message authentication code (MAC). Dopodiché, l'LPA mostra l'operatore all'utente, che dovrà stabilire se è corretto: se sì, l'LPA inoltra tutte le informazioni all'eUICC il quale, dopo aver derivato a sua volta le chiavi di sessione k, k', dovrà decriptare il profilo P, che risulterà finalmente essere utilizzabile.
\end{itemize}
\begin{figure}
\includegraphics[width=\linewidth]{default-server.png}
\caption{Approccio default server per le fasi di profile ordering e download initialization.}
\label{fig:default-server}
\end{figure}
\begin{figure}
\includegraphics[width=\linewidth]{activation-code.png}
\caption{Approccio Activation Code per le fasi di profile ordering e download initialization.}
\label{fig:activation-code}
\end{figure}

\subsection{Ciclo di vita dei profili in SM-DP+}
Prima di approfondire più nel dettaglio l'interazione in RSP, è bene illustrare il ciclo di vita dei profili eSIM.\\
La tabella \ref{tab:profile-states} tratta da \cite{GSMA-docs-new}\cite{GSMA-docs-old} fornisce un elenco degli stati in cui ciascun profilo eSIM può trovarsi nell'arco della sua esistenza.
\begin{table}[h!]
\begin{center}
\captionsetup{skip=4pt}
\caption{Stati dei profili eSIM}
\label{tab:profile-states}
\begin{tabularx}{\textwidth}{|c|X|}
\hline
\textbf{Nome} & \textbf{Descrizione}\\
\hline
Available & Il profilo è disponibile nell’inventory del server SM-DP+.\\
\hline
Allocated & Il profilo è riservato per il download senza essere linkato a un EID (eUICC ID).\\
\hline
Linked & Il profilo è riservato per il download ed è linkato a un EID.\\
\hline
Confirmed & Il profilo è riservato per il download (che sia esso linkato o non linkato a un EID) col Matching ID (i.e. il codice che identifica la transazione di download) e il codice di conferma (i.e. il codice che deve essere inserito dall'end user), se richiesti.\\
\hline
Released & Il profilo è pronto per il download e l’installazione dopo che l’operatore ha effettuato la configurazione di rete.\\
\hline
Downloaded & Il profilo è stato consegnato all’LPA (i.e. è stato scaricato).\\
\hline
Installed & Il profilo è stato installato sull’eUICC con successo.\\
\hline
Error & Il profilo non è stato installato a causa di una condizione di errore.\\
\hline
Unavailable & Il profilo non può essere più riutilizzato da SM-DP+.\\
\hline
\end{tabularx}
\end{center}
\end{table}
Nelle figure \ref{fig:profile-states1}, \ref{fig:profile-states2} tratte da \cite{GSMA-docs-new}\cite{GSMA-docs-old}, invece, sono mostrati due diagrammi a stati finiti che illustrano per bene il ciclo di vita dei profili in SM-DP+.
\begin{figure}
\includegraphics[width=\linewidth]{profile-states1.png}
\caption{Primo diagramma a stati per i profili eSIM.}
\label{fig:profile-states1}
\end{figure}
\begin{figure}
\includegraphics[width=\linewidth]{profile-states2.png}
\caption{Secondo diagramma a stati per i profili eSIM.}
\label{fig:profile-states2}
\end{figure}
\\
\\Con riferimento alla figura \ref{fig:profile-states1}, a partire dallo stato Available:
\begin{itemize}
\item Si passa allo stato Allocated se si effettua l'ordine di download senza specificare l'EID.
\item Si passa allo stato Linked se si effettua l'ordine di download specificando l'EID.
\end{itemize}
A partire dallo stato Allocated:
\begin{itemize}
\item Si passa allo stato Confirmed se si conferma l'ordine di download con releaseFlag=false.
\item Si passa allo stato Released se si conferma l'ordine di download con releaseFlag=true.
\end{itemize}
A partire dallo stato Linked:
\begin{itemize}
\item Si passa allo stato Confirmed se si conferma l'ordine di download con releaseFlag=false.
\item Si passa allo stato Released se si conferma l'ordine di download con releaseFlag=true.
\end{itemize}
A partire dallo stato Confirmed:
\begin{itemize}
\item Si passa allo stato Released se si effettua il rilascio del profilo in modo tale che sia effettivamente pronto per il download e l'installazione.
\end{itemize}
A partire dallo stato Released:
\begin{itemize}
\item Si passa allo stato Downloaded se il profilo viene consegnato all'LPA con successo.
\item Si passa allo stato Error se si ha un errore nel consegnare il profilo all'LPA.
\end{itemize}
A partire dallo stato Downloaded:
\begin{itemize}
\item Si passa allo stato Installed se il profilo viene installato sull'eUICC con successo.
\item Si passa allo stato Error se si ha un errore nell'installare il profilo sull'eUICC.
\end{itemize}
Con riferimento alla figura \ref{fig:profile-states2}, a partire dagli stati Allocated / Linked / Confirmed / Released:
\begin{itemize}
\item Si torna allo stato Available se l'ordine di download viene annullato con finalProfileStatusIndicator=Available.
\item Si passa allo stato Unavailable se l'ordine di download viene annullato con finalProfileStatusIndicator=Unavailable.
\end{itemize}
A partire dallo stato Error:
\begin{itemize}
\item Si torna allo stato Available con una transizione automatica oppure se l'ordine di download viene annullato con finalProfileStatusIndicator=Available.
\item Si passa allo stato Unavailable con una transizione automatica oppure se l'ordine di download viene annullato con finalProfileStatusIndicator=Unavailable.
\end{itemize}

\subsection{Dettagli sulla Common Mutual Authentication}\label{sec:mutual-auth}
La figura \ref{fig:common-mutual-auth} ripresa da \cite{GSMA-docs-new} illustra tutti i messaggi che il server SM-DP+, l'LPA e l'eUICC si scambiano tra loro e le operazioni che queste tre entità svolgono durante la procedura di Common Mutual Authentication. Tutti i relativi dettagli \cite{GSMA-docs-new} sono spiegati nelle sottosezioni successive. Si osservi che l'intero meccanismo resta valido e invariato se si ha un server SM-DS al posto del server SM-DP+.
\begin{figure}
\includegraphics[width=\linewidth]{common-mutual-auth.png}
\caption{Sequence diagram che descrive la Common Mutual Authentication.}
\label{fig:common-mutual-auth}
\end{figure}

\subsubsection{Condizioni iniziali}
\begin{itemize}
\item Il server SM-DP+ è dotato di:
\begin{itemize}[itemsep=0pt]
\item certificato CERT.DPauth.SIG;
\item chiave privata SK.DPauth.SIG;
\item certificato del CI (CERT.CI.SIG);
\item certificato TLS CERT.DP.TLS;
\item chiave privata TLS SK.DP.TLS;
\item certificati di SM-DP+ intermediari, se esistenti (CERT.DPSubCA.SIG).
\end{itemize}
\item L'eUICC, d'altra parte, è dotato di:
\begin{itemize}[itemsep=0pt]
\item certificato CERT.EUICC.SIG;
\item chiave privata SK.EUICC.SIG;
\item certificato dell'EUM (CERT.EUM.SIG);
\item certificati di CI subordinati, se esistenti (CERT.CISubCA.SIG);
\item certificati di EUM subordinati, se esistenti (CERT.EUMSubCA.SIG);
\item chiave pubblica del CI (PK.CI.SIG).
\end{itemize}
\end{itemize}

\subsubsection{Procedimento}
\begin{enumerate}
\item Il primo step, che è opzionale, si articola in tre punti:
\begin{enumerate}[label=\alph*)]
\item Se non lo aveva già fatto in un momento precedente, l'LPA richiede le informazioni dell'eUICC (i.e. Specification Version Number e identificatori delle chiavi pubbliche del CI che possono essere utilizzate per autenticare il server e per firmare i propri dati) identificate dalla variabile euiccInfo1.
\item L'eUICC restituisce euiccInfo1 all'LPA.
\item Se esiste una restrizione sulle chiavi pubbliche consentite del CI, l'LPA crea una nuova istanza di euiccInfo1 senza le chiavi pubbliche non compatibili del CI. Se dopo questa operazione rimane una lista vuota di chiavi pubbliche, l'LPA informa l'end user che la procedura di mutua autenticazione deve terminare.
\end{enumerate}
\item L'LPA richiede all'eUICC una challenge (euiccChallenge).
\item L'eUICC genera la challenge che successivamente dovrà essere firmata dal server SM-DP+ per l'autenticazione del server stesso.
\item L'eUICC restituisce la challenge all'LPA.
\item L'LPA stabilisce una nuova connessione HTTPS col server SM-DP+. Il setup della sessione TLS, poiché non può riutilizzare le chiavi da una sessione precedente, deve prevedere un nuovo key exchange, che avviene nella fase del TLS handshake. Qui il server fornisce all'LPA un certificato CERT.DP.TLS e l'LPA deve verificarne la validità; se l'LPA non riesce a effettuare la verifica, il server deve inviargli un certificato CERT.DP.TLS differente, e così via, fin tanto che l'LPA non sarà riuscito a validare un certificato oppure il numero di tentativi per ritrasmettere il certificato non avrà raggiunto il limite massimo. In questo secondo caso l'LPA interrompe la procedura di Common Mutual Authentication.
\item L'LPA invoca la funzione \textit{InitiateAuthentication} passandovi come parametri le proprie capability, euiccChallenge, euiccInfo1 e l'indirizzo SM-DP+, il quale viene usato dall'LPA per accedere al server.
\item Il server SM-DP+ esegue le seguenti operazioni:
\begin{itemize}[itemsep=0pt]
\item Verificare che l'indirizzo SM-DP+ inviato dall'LPA sia valido.
\item Verificare il contenuto della variabile euiccInfo1, incluse le chiavi pubbliche del CI, tra cui deve essercene almeno una che può essere accettata dal server stesso.
\end{itemize}
Se anche solo uno di questi controlli non va a buon fine, il server restituisce una condizione di errore e l'LPA interrompe la procedura di Common Mutual Authentication.
\item Il server SM-DP+ esegue le seguenti altre operazioni:
\begin{itemize}[itemsep=0pt]
\item Generare un ID di transazione che serve per identificare la sessione RSP.
\item Generare una challenge (serverChallenge) che dovrà essere firmata dall'eUICC per l'autenticazione dell'eUICC stesso.
\item Generare una struttura dati serverSigned1 a partire dall'ID di transazione, da euiccChallenge, da serverChallenge, dall'indirizzo SM-DP+ ed eventualmente dal contesto di sessione.
\item Calcolare la serverSignature1 a partire da serverSigned1, utilizzando la chiave privata SK.DPauth.SIG.
\item Se sia l'eUICC che l'LPA prevedono il crlStaplingV3Support, che è la funzionalità che permette al server di fornire una Certificate Revocation List (CRL), il server deve anche recuperare la CRL per ciascun certificato che abbia l'estensione CRLDistributionPoints (i.e. per ciascun certificato che dà informazioni su dove è possibile ottenere una CRL).
\end{itemize}
\item Il server SM-DP+ restituisce all'LPA alcune informazioni, tra cui l'ID della transazione, serverSigned1, serverSignature1, euiccCIPKToBeUsed (che dovrà corrispondere alla chiave pubblica del CI accettata tra quelle proposte dall'eUICC), il certificato CERT.DPauth.SIG, eventuali altri certificati ed eventualmente la CRL.
\item L'LPA esegue le seguenti operazioni:
\begin{itemize}[itemsep=0pt]
\item Verificare l'OID, che è l'Object Identifier del server SM-DP+ (se fornito in precedenza).
\item Verificare che l'indirizzo SM-DP+ restituito dal server (incapsulato in serverSigned1) matchi con l'indirizzo SM-DP+ che l'LPA aveva inviato allo step (6).
\item Verificare che la chiave pubblica associata al certificato CERT.DPauth.SIG sia inclusa in euiccInfo1.
\item Effettuare altre verifiche sui certificati che non verranno approfondite in questa sede.
\end{itemize}
Se anche solo uno di questi controlli non va a buon fine, l'LPA interrompe la procedura di Common Mutual Authentication. In caso contrario, procede col generare la struttura dati ctxParams1, che dovrà essere inviata all'eUICC affinché venga poi inclusa tra i dati firmati.
\item L'LPA invoca la funzione \textit{AuthenticateServer} passandovi come parametri serverSigned1, serverSignature1, euiccCiIdToBeUsed, il certificato CERT.DPauth.SIG, eventuali altri certificati, ctxParams1 ed eventualmente la CRL.
\item L'eUICC esegue le seguenti operazioni:
\begin{itemize}[itemsep=0pt]
\item Verificare il certificato CERT.DPauth.SIG e altri eventuali certificati nella catena.
\item Verificare serverSignature1.
\item Verificare che l'euiccChallenge contenuta in serverSigned1 matchi con la challenge generata dall'eUICC stessa allo step (3).
\item Verificare che la chiave pubblica del CI sia effettivamente supportata.
\item Se il contesto di sessione prevede il crlStaplingV3Support, l'eUICC deve anche verificare la validità della CRL e assicurarsi che nessun certificato all'interno della certificate chain sia stato revocato.
\end{itemize}
Se anche solo uno di questi controlli non va a buon fine, la procedura di Common Mutual Authentication deve essere interrotta. In caso contrario, il server SM-DP+ risulta autenticato all'eUICC.
\item L'eUICC esegue le seguenti altre operazioni:
\begin{itemize}[itemsep=0pt]
\item Generare la struttura dati euiccSigned1 a partire dall'ID di transazione, dall'indirizzo del server SM-DP+, da serverChallenge, da euiccInfo2 e da ctxParams1, dove euiccInfo2 è un sovrainsieme di euiccInfo1 e comprende le informazioni complete dell'eUICC. Si noti che euiccChallenge non è incluso in euiccSigned1.
\item Calcolare la euiccSignature1 a partire da euiccSigned1, utilizzando la chiave privata SK.EUICC.SIG.
\end{itemize}
\item L'eUICC restituisce all'LPA alcune informazioni, tra cui euiccSigned1, euiccSignature1 e la catena di certificati dell'eUICC.
\item L'LPA invoca la funzione \textit{AuthenticateClient} passandovi come parametri euiccSigned1, euiccSignature1 e la catena di certificati dell'eUICC.
\item Il server SM-DP+ esegue le seguenti operazioni:
\begin{itemize}[itemsep=0pt]
\item Verificare che l'ID di transazione contenuto in euiccSigned1 corrisponda a quello comunicato dal server stesso allo step (9).
\item Verificare che il certificato root della certificate chain comunicata dall'eUICC corrisponda con quella selezionata dal server stesso (euiccCIPKToBeUsed) durante l'esecuzione della funzione \textit{InitiateAuthentication}.
\item Verificare che la certificate chain dell'eUICC sia valida.
\item Verificare euiccSignature1.
\item Verificare che la serverChallenge contenuta in euiccSigned1 matchi con la challenge generata dal server stesso allo step (8).
\end{itemize}
Se anche solo uno di questi controlli non va a buon fine, il server restituisce una condizione di errore e l'LPA interrompe la procedura di Common Mutual Authentication. In caso contrario, l'eUICC risulta autenticato al server SM-DP+.
\end{enumerate}
In definitiva, in questa sezione è emerso come l'LPA svolga sì il ruolo di relay nell'interazione tra server SM-DP+ ed eUICC, ma svolge anche delle importanti funzioni di sicurezza. Infatti, com'è stato già menzionato, si occupa anzitutto di verificare che il certificato CERT.DP.TLS sia valido e, in un secondo momento, effettua altri controlli sull'OID, sull'indirizzo del server SM-DP+, sulla chiave pubblica associata al certificato CERT.DPauth.SIG e sulla certificate chain del server nello specifico. Tale caratteristica implica la necessità di considerare l'applicazione LPA come un'entità trusted, il che non è fortemente indicato dal punto di vista della sicurezza.

\subsection{Dettagli su download e installazione dei profili}\label{sec:down-install}
La figura \ref{fig:download-install} ripresa da \cite{GSMA-docs-new} illustra tutti i messaggi che l'operatore, il server SM-DP+, l'LPA e l'eUICC si scambiano tra loro e le operazioni che queste quattro entità svolgono durante la procedura di download e installazione dei profili. Tutti i relativi dettagli \cite{GSMA-docs-new} sono spiegati nelle sottosezioni successive. Anche qui l'intero meccanismo resta valido se si ha un server SM-DS al posto del server SM-DP+, a meno di variazioni di poco conto.
\begin{figure}
\includegraphics[width=\linewidth]{download-install.png}
\caption{Sequence diagram che descrive il download e l'installazione dei profili.}
\label{fig:download-install}
\end{figure}

\subsubsection{Condizioni iniziali}
\begin{itemize}
\item L'applicazione LPA potrebbe aver recuperato l'indirizzo del server SM-DP+ e l'identificativo della chiave pubblica del CI Root consentita; se tale identificativo viene effettivamente recuperato a partire dall'eUICC, allora l'LPA deve restringere l'insieme degli identificativi delle chiavi pubbliche dei CI Root compatibili a quel valore.
\item Per ogni profilo nello stato Released il server SM-DP+ mantiene un contatore dei tentativi di download di quel profilo e un contatore dei tentativi di immissione del codice di conferma. Di fatto, il server deve limitare il valore di questi due contatori.
\item Se è richiesto l'Activation Code per il download e l'installazione del profilo, l'end user deve averlo già immesso all'LPA.
\end{itemize}

\subsubsection{Procedimento}
\begin{enumerate}
\item Se non lo ha già fatto in precedenza e se è necessario, l'applicazione LPA effettua il parsing dell'Activation Code: così ottiene l'indirizzo del server SM-DP+ e, opzionalmente, l'OID del server SM-DP+ e l'identificativo della chiave pubblica del CI Root.
\item Viene eseguita la procedura di Common Mutual Authentication definita nella sezione \ref{sec:mutual-auth}. In particolare, nel messaggio \textit{AuthenticateClient}, l'LPA deve specificare il Matching ID, che è l'identificatore della transazione di download corrente e, dunque, individua esattamente il profilo che si vuole installare.
\item Il server SM-DP+ esegue le seguenti operazioni:
\begin{itemize}[itemsep=0pt]
\item Verificare che esista un profilo correlato al Matching ID ricevuto in \textit{AuthenticateClient}.
\item Se l'ordinazione di download del profilo è già linkato a un EID, verificare che quest'ultimo corrisponda all'EID dell'eUICC appena autenticato.
\item Verificare che il profilo sia nello stato Released o, nel caso di retry a seguito del fallimento di un'installazione precedente, nello stato Downloaded.
\end{itemize}
Se anche solo uno di questi controlli non va a buon fine, la procedura di download e installazione del profilo deve essere interrotta. In caso contrario, il server SM-DP+ deve procedere con le seguenti altre azioni:
\begin{itemize}[itemsep=0pt]
\item Incrementare il contatore dei tentativi di download del profilo target. Se il numero massimo di tentativi viene sforato, il server deve notificare l'operatore del fallimento del download e l'intera procedura deve terminare.
\item Effettuare i check di idoneità opportuni.
\end{itemize}
\item Il server SM-DP+ notifica l'operatore con l'esito dei check di validità mediante la funzione \textit{handleNotification} (step opzionale). La comunicazione tra server e operatore è protetta dall'uso di un pairwise secure channel, come nell'interazione tra eUICC e LPA; anche qui il protocollo più utilizzato è TLS ma non viene imposto dallo standard di GSMA. Di conseguenza, potrebbe essere nuovamente utile stabilire empiricamente se nel canale di comunicazione tra server SM-DP+ e operatore viene utilizzato TLS o meno.
\item Se i check di idoneità falliscono, allora il server SM-DP+ esegue le seguenti operazioni:
\begin{itemize}[itemsep=0pt]
\item Portare il profilo target allo stato Error.
\item Restituire uno status di errore all'LPA in modo tale che l'intera procedura termini.
\end{itemize}
In caso contrario, il server SM-DP+ esegue le seguenti altre operazioni:
\begin{itemize}[itemsep=0pt]
\item Determinare se è richiesto un codice di conferma (Confirmation Code) per l'ordinazione pendente.
\item Generare una struttura dati smdpSigned2 che contiene le proprie informazioni.
\item Calcolare smdpSignature2, che è un fingerprint ottenuto da smdpSigned2 ed euiccSignature1, dove euiccSignature1 è un'informazione che è stata scambiata durante la fase di Common Mutual Authetication.
\item Generare i metadati del profilo target.
\end{itemize}
\item Il server SM-DP+ fornisce all'LPA la risposta ad \textit{authenticateClient} (funzione invocata durante la procedura di Common Mutual Authentication), che comprende transactionId, i metadati del profilo, smdpSigned2, smdpSignature2 e CERT.DPpb.SIG.
\item L'LPA verifica se il profilo può essere effettivamente installato. Per far ciò, deve ricorrere a un'apposita struttura dati detta Rules Authorisation Table (RAT) e/o alla lista dei profili installati. Se non dispone già di queste informazioni, deve richiederle all'eUICC invocando le funzioni \textit{getRAT} e/o \textit{getProfilesInfo}. Informazioni più dettagliate sulla RAT e, più in generale, sulla gestione delle policy dei profili, sono riportate nella sezione \ref{sec:profile-policy}.
\item Se necessario, l'LPA richiede all'end user di inserire il codice di conferma che l'operatore gli aveva fornito. In caso contrario, l'LPA richiede una conferma semplice (Simple Confirmation) sul download del profilo, che può prevedere poche semplici opzioni di risposta come 'Yes', 'No', 'Not now'. Se l'end user non inserisce correttamente il codice di conferma o non risponde positivamente alla conferma semplice, si procede con la cancellazione della sessione (Common Cancel Session Procedure, i cui dettagli sono riportati nella sezione \ref{sec:common-cancel-session}); altrimenti, l'installazione del profilo può completarsi con successo mediante la sotto-procedura di Download Confirmation, che verrà illustrata qui di seguito.
\end{enumerate}
%In definitiva, sembra che il modo più semplice per bucare la procedura di download e installazione dei profili sia sfruttare eventuali vulnerabilità presenti nella procedura di Common Mutual Authentication.

\subsection{Sotto-procedura di Download Confirmation}\label{sec:down-confirm}
La figura \ref{fig:download-confirmation} ripresa da \cite{GSMA-docs-new} illustra tutti i messaggi che l'operatore, il server SM-DP+, l'LPA e l'eUICC si scambiano tra loro e le operazioni che queste quattro entità svolgono durante la sotto-procedura di Download Confirmation, che conclude la procedura di download e installazione dei profili introdotta poc'anzi. Tutti i relativi dettagli \cite{GSMA-docs-new} sono spiegati nelle sottosezioni successive. Ancora una volta i passaggi rimangono gli stessi se si ha un server SM-DS al posto del server SM-DP+.
\begin{figure}
\includegraphics[width=\linewidth]{download-confirmation.png}
\caption{Sequence diagram che descrive la sotto-procedura di Download Confirmation.}
\label{fig:download-confirmation}
\end{figure}

\subsubsection{Condizioni iniziali}
\begin{itemize}
\item L'end user ha consentito con successo il download del profilo target.
\end{itemize}

\subsubsection{Procedimento}
\begin{enumerate}
\item L'LPA invoca la funzione \textit{prepareDownload} passandovi come parametri le informazioni precedentemente ricevute del server SM-DP+ (come smdpSigned2, smdpSignature2 e il certificato CERT.DPpb.SIG) e opzionalmente l'hash del codice di conferma (Confirmation Code).
\item L'eUICC esegue le seguenti operazioni:
\begin{itemize}[itemsep=0pt]
\item Verificare che CERT.DPpb.SIG sia un certificato valido.
\item Verificare che CERT.DPauth.SIG e CERT.DPpb.SIG appartengano alla medesima entità e siano state rilasciate dalla medesima CA.
\item Verificare smdpSignature2.
\item Verificare che l'ID di transazione contenuto in smdpSigned2 corrisponda con l'ID di transazione che identifica la sessione RSP corrente.
\end{itemize}
Se anche solo uno di questi controlli non va a buon fine, l'eUICC deve restituire uno stato di errore e la procedura di download e installazione del profilo deve essere interrotta. 
\item L'eUICC prosegue con le seguenti altre azioni:
\begin{itemize}[itemsep=0pt]
\item Generare una coppia di chiavi one-time (otPK.EUICC.KA, otSK.EUICC.KA).
\item Generare la struttura dati euiccSigned2, che comprende l'ID di transazione, otPK.EUICC.KA ed eventualmente l'hash del Confirmation Code.
\item Calcolare euiccSignature2.
\end{itemize}
\item L'eUICC restituisce all'LPA la risposta a \textit{prepareDownload}.
\item L'LPA invoca la funzione \textit{getBoundProfilePackage} passandovi come parametri euiccSigned2 ed euiccSignature2.
\item Il server SM-DP+ verifica euiccSignature2 e stabilisce se è richiesto il codice di conferma: se sì, si procede con lo step (7), altrimenti si passa allo step (9).
\item Il server SM-DP+ esegue le seguenti operazioni:
\begin{itemize}[itemsep=0pt]
\item Recuperare l'hash del Confirmation Code ottenuto dalla funzione \textit{confirmOrder} relativa alle fasi di profile ordering \& download initialization, e calcolarne il valore atteso come SHA256(SHA256(Confirmation Code) | transactionID).
\item Verificare che l'hash del Confirmation Code ricevuto all'interno di euiccSigned2 corrisponda col valore hash atteso.
\item Se la verifica dell'hash del Confirmation Code fallisce, incrementare di un'unità il contatore dei tentativi di immissione del codice di attivazione e verificare che tale contatore non sfori il numero massimo ammissibile.
\end{itemize}
Se anche solo uno di questi controlli non va a buon fine, il server SM-DP+ deve restituire uno stato di errore e la procedura di download e installazione del profilo deve essere interrotta. Più precisamente, se fallisce il controllo sul contatore dei tentativi di immissione del codice di attivazione, è necessario anche impostare il profilo target nello stato Error e passare allo step (8) prima di interrompere la procedura. Se invece i controlli vanno tutti a buon fine, si passa allo step (9).
\item Se il numero di tentativi di immissione del codice di attivazione supera il massimo stabilito, il server SM-DP+ informa l'operatore del fallimento mediante la funzione \textit{handleNotification}; dopodiché, la procedura di download e installazione del profilo viene interrotta.
\item Il server SM-DP+ genera un Bound Profile Package.
\item Il server SM-DP+ può opzionalmente notificare l'operatore del completamento del download del profilo mediante la funzione \textit{handleNotification}.
\item Il server SM-DP+ invia all'LPA la risposta a \textit{getBoundProfilePackage} e imposta lo stato del profilo target a Downloaded.
\item Se l'LPA ha già trattato la struttura dati Profile Metadata in precedenza, allora deve assicurarsi che le informazioni contenute nel Bound Profile Package (BPP) appena ricevuto le informazioni relative ai metadati del profilo siano rimaste coerenti; se non dovessero esserlo, deve avviare la Common Cancel Session Procedure (\ref{sec:common-cancel-session}) per cancellare la sessione corrente.
\item L'LPA e l'eUICC si scambiano una serie di messaggi per effettuare opportuni controlli e per portare a termine l'installazione del profilo all'interno dell'eUICC.
\item L'LPA invoca la funzione \textit{handleNotification} con lo scopo di consegnare il risultato dell'installazione del profilo al server SM-DP+, il quale risponderà con un semplice acknowledgement.
\item Il server SM-DP+ esegue le seguenti operazioni:
\begin{itemize}[itemsep=0pt]
\item Recuperare l'ordinazione di download del profilo target identificata dall'ID di transazione. Se viene ricevuto un ID di transazione ignoto, bisogna interrompere la procedura.
\item Chiudere l'ordinazione di download del profilo target e impostare il relativo stato a Installed o Error in base a quanto indicato dal risultato dell'installazione del profilo.
\end{itemize}
\item Eventualmente il server SM-DP+ invoca ancora una volta la funzione \textit{handleNotification} specificando l'esito dell'installazione del profilo sull'eUICC.
\item Se oltre al server SM-DP+ è coinvolto anche un SM-DS, allora il server SM-DP+ esegue la procedura di eliminazione dell'evento dall'SM-DS.
\item Alla ricezione dell'acknowledgement da parte del server SM-DP+, l'LPA invoca la funzione \textit{removeNotificationFromList} sull'eUICC.
\item L'eUICC cancella il risultato dell'installazione del profilo dalla propria memoria non volatile.
\item Se l'LPA aveva ricevuto rpmPending nel messaggio di risposta ad \textit{authenticateClient}, allora dovrebbe iniziare a una sessione RSP aggiuntiva col server SM-DP+, settando l'operationType a RPM (Remote Profile Management).
\end{enumerate}

\subsection{Cambio di dispositivo}\label{sec:device-change}
La figura \ref{fig:device-change} ripresa da \cite{GSMA-docs-new} illustra tutti i messaggi che l'end user, l'LPA installata nel vecchio dispositivo, l'eUICC presente nel vecchio dispositivo, il server SM-DP+, l'LPA installata nel nuovo dispositivo, l'eUICC presente nel nuovo dispositivo e l'operatore (qui indicato come Service Provider) si scambiano tra loro e le operazioni che queste sette entità svolgono durante la procedura di trasferimento di un profilo dovuto al cambio di dispositivo. Tutti i relativi dettagli \cite{GSMA-docs-new} sono spiegati nelle sottosezioni successive.
\begin{figure}
\includegraphics[width=\linewidth]{device-change.png}
\caption{Sequence diagram che descrive il trasferimento di un profilo.}
\label{fig:device-change}
\end{figure}

\subsubsection{Condizioni iniziali}
\begin{itemize}
\item Il Service Provider ha fornito al server SM-DP+ la configurazione e altre informazioni rilevanti per il cambio di dispositivo. Tutti questi dati devono essere contenuti anche nel profilo all'interno del vecchio dispositivo.
\item L'end user possiede sia un vecchio dispositivo contenente un profilo, sia un nuovo dispositivo.
\item L'eUICC e l'LPA del vecchio dispositivo supportano il cambio di dispositivo.
\end{itemize}

\subsubsection{Procedimento}
\begin{enumerate}
\item L'end user dà inizio all'operazione di cambio di dispositivo a partire dall'LPA del vecchio dispositivo e seleziona il profilo da installare nel nuovo dispositivo.
\item L'LPA del vecchio dispositivo recupera DeviceChangeConfiguration dai metadati del profilo. Se DeviceChangeConfiguration indica requestToDp, allora si passa allo step (3); se invece DeviceChangeConfiguration indica usingStoredAc, allora si passa allo step (17).
\item L'LPA del vecchio dispositivo determina l'indirizzo del server SM-DP+ a partire da DeviceChangeConfiguration.
\item Se DeviceChangeConfiguration indica che è richiesto l'EID e/o la TAC (Type Allocation Code, che è un codice a 8 cifre univoco per il dispositivo) del nuovo dispositivo, allora l'LPA del vecchio dispositivo recupera l'EID e/o la TAC dal nuovo dispositivo.
\item L'LPA del vecchio dispositivo dà inizio alla procedura di Common Mutual Authentication definita nella sezione \ref{sec:mutual-auth}.
\item Se la funzione \textit{handleDeviceChangeRequest} è configurata nel Service Provider, allora il server SM-DP+ la invoca passandovi come parametri l'ICCID (Integrated Circuit Card ID, che è l'identificativo del profilo) ed eventualmente l'EID e la TAC del nuovo dispositivo. D'altro canto, però, se il server SM-DP+ non supporta il cambio di dispositivo o il cambio di dispositivo non è consentito per il profilo, allora il server risponde con un messaggio di errore e la procedura termina.
\item Il Service Provider risponde al server SM-DP+ col booleano isNewProfileRequired e, opzionalmente, un messaggio di Service Provider per il cambio di dispositivo e un codice di conferma.
\item Se la funzione \textit{handleNotification} è configurata nel Service Provider, il server SM-DP+ la invoca per notificare il Service Provider riguardo la richiesta del cambio di dispositivo.
\item Il server SM-DP+ restituisce all'LPA del vecchio dispositivo la risposta ad \textit{authenticateClient} che comprende transactionId (l'identificativo della transazione), smdpSigned4 (informazioni del server SM-DP+), smdpSignature4 (signature di smdpSigned4) ed eventualmente il messaggio di Service Provider per il cambio di dispositivo.
\item Se il booleano isNewProfileRequired fornito dal Service Provider nello step (7) vale TRUE, allora il Service Provider esegue il Download Preparation Process (processo di preparazione del download) e opzionalmente il Subscription Activation Process (processo di attivazione della sottoscrizione al profilo).
\item L'LPA del vecchio dispositivo effettua una richiesta di conferma per il cambio di dispositivo. In particolare, se il Service Provider nello step (7) ha fornito il messaggio di Service Provider per il cambio di dispositivo, quest'ultimo viene presentato all'end user. Inoltre, se smdpSigned4 ha il campo ccRequiredFlag pari a TRUE, allora l'LPA del vecchio dispositivo chiede all'end user di inserire il codice di conferma fornito dal Service Provider nello step (7). Se l'end user non inserisce correttamente il codice di conferma entro un timeout prestabilito, si procede con la cancellazione della sessione (Common Cancel Session Procedure, \ref{sec:common-cancel-session}).
\item L'LPA del vecchio dispositivo invoca la funzione \textit{prepareDeviceChange} passandovi come parametri smdpSigned4, smdpSignature4 ed eventualmente l'hash del codice di conferma. Quest'ultimo viene calcolato come SHA256(SHA256(Confirmation Code) | transactionID).
\item L'LPA del vecchio dispositivo invoca la funzione \textit{confirmDeviceChange} passandovi come parametri transactionId e prepareDeviceChangeResponse.
\item Se il booleano isNewProfileRequired fornito dal Service Provider nello step (7) vale TRUE, allora il server SM-DP+ invoca la funzione \textit{handleNotification} per notificare il Service Provider riguardo l'esito della conferma del cambio di dispositivo da parte dell'end user. Se l'end user ha accettato il cambio di dispositivo, allora si passa allo step (15); altrimenti la procedura termina.
\item Se il booleano isNewProfileRequired vale FALSE, allora il server SM-DP+ prepara il profilo per il download e il Matching ID associato al profilo. Inoltre, se nello step (5) è stato fornito un EID, allora il server lo collega al download del profilo preparato. Infine, il server invoca la funzione \textit{handleNotification} per notificare il Service Provider riguardo l'esito della preparazione del profilo.
\item Il server SM-DP+ restituisce all'LPA del vecchio dispositivo la risposta a \textit{confirmDeviceChange} che comprende l'esito del cambio di dispositivo. Se tale risposta contiene encryptedDeviceChangeData, l'LPA del vecchio dispositivo disabilita il profilo target. Dopodiché l'LPA verifica la signature del server SM-DP+ invocando la funzione \textit{verifyDeviceChange}.
\item Se previsto, l'LPA del vecchio dispositivo elimina il profilo target dall'eUICC del vecchio dispositivo. Inoltre, se DeviceChangeConfiguration indica requestToDp e il server SM-DP+ nello step (16) ha indicato di supportare la recovery dei profili eliminati, allora l'LPA del vecchio dispositivo dovrebbe memorizzare alcune informazioni del profilo eliminato, come l'ICCID. Se invece l'eliminazione del profilo non è richiesta, si passa direttamente allo step (19).
\item L'LPA del vecchio dispositivo invia al server SM-DP+ la notifica di eliminazione del profilo. Se l'invio della notifica fallisce, la procedura termina.
\item L'LPA del vecchio dispositivo genera il codice di attivazione (Activation Code) e lo fornisce all'LPA del nuovo dispositivo. Inoltre, dovrebbe fornire al nuovo dispositivo lo stato attuale del profilo in modo tale da consentire all'LPA del nuovo dispositivo di ripristinare correttamente tale stato.
\item Il profilo viene scaricato nel nuovo dispositivo a partire dal server SM-DP+ mediante la procedura di download e installazione del profilo, che è stata definita nella sezione \ref{sec:down-install}.
\end{enumerate}

\subsection{Dettagli sulla Common Cancel Session Procedure}\label{sec:common-cancel-session}
La figura  \ref{fig:common-cancel-session} ripresa da \cite{GSMA-docs-new} illustra tutti i messaggi che l'operatore, il server SM-DP+, l'LPA e l'eUICC si scambiano tra loro e le operazioni che queste quattro entità svolgono durante la Common Cancel Session Procedure, che è una particolare procedura avviata dall'LPA con lo scopo di interrompere la sessione RSP corrente a seguito di una condizione di errore riscontrata dall'LPA stessa. Tutti i dettagli sul procedimento \cite{GSMA-docs-new} sono spiegati nella sottosezione successiva.
\begin{figure}
\includegraphics[width=\linewidth]{common-cancel-session.png}
\caption{Sequence diagram che descrive la Common Cancel Session Procedure.}
\label{fig:common-cancel-session}
\end{figure}

\subsubsection{Procedimento}
\begin{enumerate}
\item L'LPA invoca la funzione \textit{cancelSession} sull'eUICC, passandovi come parametri l'ID di transazione e la reason, ovvero il motivo per cui è stata avviata la Common Cancel Session Procedure.
\item L'eUICC esegue le seguenti operazioni:
\begin{itemize}[itemsep=0pt]
\item Generare la struttura dati euiccCancelSessionSigned che comprende l'ID di transazione e la reason, entrambi forniti dall'LPA.
\item Calcolare euiccCancelSessionSignature a partire da euiccCancelSessionSigned utilizzando la chiave privata SK.EUICC.SIG associata all'euiccCiPKIdToBeUsed ricevuto all'inizio della procedura di Common Mutual Authentication.
\end{itemize}
\item L'eUICC restituisce euiccCancelSessionSigned ed euiccCancelSessionSignature all'LPA. Se la reason è sessionAborted, allora l'LPA dovrà ignorare tale messaggio di risposta e interrompere la procedura.
\item Se la connessione HTTPS tra LPA e server SM-DP+ non è più attiva, allora l'LPA stabilisce una nuova connessione HTTPS col server come descritto nella procedura di Common Mutual Authentication.
\item L'LPA invoca la funzione \textit{cancelSession} anche sul server SM-DP+, passandovi come parametri l'ID della transazione, euiccCancelSessionSigned ed euiccCancelSessionSignature.
\item Il server SM-DP+ esegue le seguenti operazioni:
\begin{itemize}[itemsep=0pt]
\item Recuperare la sessione RSP corrente identificata dall'ID di transazione appena ricevuto.
\item Verificare euiccCancelSessionSignature.
\item Verificare che l'OID ricevuto sia uguale a quello contenuto in CERT.DPauth.SIG durante la procedura di Common Mutual Authentication.
\end{itemize}
Se anche solo uno di questi controlli non va a buon fine, o se l'ID di transazione ricevuto è ignoto, il server SM-DP+ deve restituire uno stato di errore e la procedura di Common Cancel Session deve essere interrotta. Inoltre, se la reason è postponed o timeout, allora il server deve mantenere l'ordinazione di download del profilo disponibile per ulteriori tentativi e passare allo step (9); in caso contrario, si procede con lo step successivo.
\item Il server SM-DP+ imposta il profilo allo stato Error. Inoltre, se un SM-DS è coinvolto nella sessione RSP, allora il server deve anche eliminare l'evento corrispondente dall'SM-DS.
\item Il server SM-DP+ può opzionalmente notificare l'operatore del fallimento del download del profilo mediante la funzione \textit{handleNotification}.
\item Il server SM-DP+ restituisce un acknowledgement all'LPA. A questo punto, la procedura termina.
\end{enumerate}

\section{Policy dei profili}\label{sec:profile-policy}
Come accennato nella sotto-sezione \ref{sec:down-install}, durante la procedura di download e installazione del profilo, è necessario anche verificare le policy del profilo per stabilire se quest'ultimo può essere installato all'interno dell'eUICC. Di conseguenza, può tornare utile illustrare nel dettaglio il meccanismo di gestione delle policy dei profili previsto in GSMA \cite{GSMA-docs-new}.\\
Le policy dei profili possono essere descritte dalle cosiddette \textbf{Profile Policy Rules} (PPR), che sono delle regole che descrivono cosa è possibile e non è possibile fare con ciascun profilo. Attualmente possono essere definite due PPR in particolare:
\begin{itemize}
\item PPR1 - 'Disabling of this Profile is not allowed'
\item PPR2 - 'Deletion of this Profile is not allowed'
\end{itemize}

\subsection{Rules Authorisation Table}
La Rules Authorisation Table (RAT) è una struttura dati che tiene traccia dell'insieme delle PPR accettabili per ciascun profilo \cite{GSMA-docs-new}. Da un punto di vista strutturale, la RAT mantiene una o più entry, ciascuna delle quali è relativa a una \textbf{Profile Policy Authorisation Rule} (PPAR). Una PPAR a sua volta è composta dalle seguenti informazioni (che costituiscono le colonne della tabella RAT):
\begin{itemize}
\item \textbf{Profile Policy Rule Identifier}: identifica una o più PPR a cui questa PPAR fa riferimento.
\item \textbf{Allowed Operators}: è una lista di operatori che possono utilizzare le PPR identificate dalla colonna Profile Policy Rule Identifier.
\item \textbf{End User Consent Required}: indica se, a seguito delle PPR identificate dalla colonna Profile Policy Rule Identifier, il profilo ha bisogno di un consenso esplicito dell'end user per essere installato.
\end{itemize}
In definitiva, la possibilità per un profilo di essere installato o meno dipende dalle eventuali PPR definite per quel profilo e dalle PPAR appartenenti alla RAT. In particolare:
\begin{itemize}
\item Se il profilo non contiene alcuna PPR, allora può essere installato senza problemi.
\item In caso contrario, è necessario verificare se nella RAT esistono delle PPAR che facciano riferimento sia alle PPR contenute nel profilo, sia all'operatore che ha rilasciato il profilo:
\begin{itemize}[itemsep=0pt]
\item Se tutte le PPR vengono coperte (da una o più PPAR) nell'ambito dell'operatore che ha rilasciato il profilo, allora si controlla la colonna End User Consent Required della RAT: i PPR per cui End User Consent Required = true hanno bisogno del consenso dell'end user affinché il relativo profilo possa essere installato nell'eUICC, e viceversa.
\item Se invece non tutte le PPR vengono coperte, non c'è la possibilità di installare il profilo nell'eUICC.
\end{itemize}
\end{itemize}
Si noti che il placeholder "*" per Allowed Operators indica "tutti gli operatori". Se esistono più PPAR valide per una medesima PPR e un medesimo operatore, fa fede quella che compare per prima all'interno della RAT.\\
Il componente dell'eUICC che si occupa di verificare se un profilo contenente una o più PPR è autorizzato dalla RAT o meno è detto \textbf{Profile Policy Enabler} (PPE) \cite{GSMA-docs-new}.

\subsubsection{Primo esempio di RAT}
La tabella \ref{tab:rat1}, tratta da \cite{GSMA-docs-new}, raffigura un primo esempio di RAT.\\
\begin{table}[h!]
\begin{center}
\captionsetup{skip=4pt}
\caption{Primo esempio di RAT}
\label{tab:rat1}
\begin{tabularx}{\textwidth}{|X|c|c|} % <-- Alignments: 2st column center and 3nd center, with vertical lines in between
\hline
\textbf{Profile Policy Rule Identifier} & \textbf{Allowed Operators} & \textbf{End User Consent Required}\\
\hline
PPR1 & OP-A & false\\
\hline
PPR2 & OP-B & false\\
\hline
PPR1, PPR2 & * & true\\
\hline
\end{tabularx}
\end{center}
\end{table}

La RAT rappresentata in \ref{tab:rat1} descrive la seguente situazione:
\begin{itemize}
\item L'operatore OP-A può usare PPR1 senza il consenso dell'end user e PPR2 col consenso dell'end user.
\item L'operatore OP-B può usare PPR1 col consenso dell'end user e PPR2 senza il consenso dell'end user.
\item Tutti gli altri operatori possono usare sia PPR1 che PPR2 col consenso dell'end user.
\end{itemize}

\subsubsection{Secondo esempio di RAT}
La tabella \ref{tab:rat2}, tratta da \cite{GSMA-docs-new}, raffigura un secondo esempio di RAT.\\
\begin{table}[h!]
\begin{center}
\captionsetup{skip=4pt}
\caption{Secondo esempio di RAT}
\label{tab:rat2}
\begin{tabularx}{\textwidth}{|X|c|c|} % <-- Alignments: 2st column center and 3nd center, with vertical lines in between
\hline
\textbf{Profile Policy Rule Identifier} & \textbf{Allowed Operators} & \textbf{End User Consent Required}\\
\hline
PPR1, PPR2 & * & true\\
\hline
\end{tabularx}
\end{center}
\end{table}

Secondo la RAT rappresentata in \ref{tab:rat2}, tutti gli operatori possono usare PPR1 e PPR2 col consenso dell'end user.

\subsubsection{Terzo esempio di RAT}
Un terzo esempio classico è dato dalla tabella RAT vuota, secondo cui nessun operatore può usare alcuna PPR.

\chapter{Analisi di sicurezza dell'eSIM}
\section{Intento dell'analisi}
L'obiettivo che è stato inizialmente fissato per la fase operativa del presente lavoro è quello di costruire un simulatore sia per il client (composto da eUICC e LPA) sia per il server SM-DP+, in modo tale da:
\begin{enumerate}
\item creare un ecosistema simulato che funzioni secondo le specifiche della GSMA;
\item far comunicare ciascun simulatore rispettivamente con server SM-DP+ reali e con eUICC + LPA reali, con l'idea di verificare come questi rispondono se vengono sollecitati in modo diverso. In tal modo, con test opportuni, è possibile mettere alla prova le entità reali per stabilire se:
\begin{itemize}[itemsep=0pt]
\item è possibile ricavare qualche informazione in più o qualche vulnerabilità;
\item implementano ogni controllo e ogni messaggio secondo le specifiche della GSMA oppure accettano un qualche pacchetto non conforme (e.g. contenente una signature o un certificato non valido).
\end{itemize}
\end{enumerate}
Per aiutarsi con l'implementazione dei simulatori, è utile avere a disposizione una traccia che rappresenti un reale scambio di messaggi tra un'applicazione LPA e un server SM-DP+. Questa, assieme alla documentazione, può rappresentare il punto di riferimento a cui affidarsi per implementare nel modo più preciso e fedele possibile lo scambio di messaggi e i check effettuati dalle due controparti.

\section{Ottenimento delle catture su Wireshark}
È stato effettuato il download di un nuovo profilo (i.e. piano tariffario) reale di Very Mobile su un apposito dispositivo mobile che supporta le eSIM. Lo scambio di messaggi che ha coinvolto l'LPA del dispositivo mobile e il server SM-DP+ di Very Mobile è stato salvato su una traccia PCAP con l'ausilio di un'apposita applicazione di cattura di pacchetti di rete. Com'è stato già detto, il traffico tra LPA e server è traffico HTTPS per cui, di base, la traccia che si ottiene mostra dei pacchetti protetti dal protocollo TLS, offuscando il contenuto dei messaggi previsti dalle specifiche di GSMA di cui si è parlato nel capitolo precedente.\\
Per questo motivo, durante la fase di cattura, è stato sfruttato un proxy relay che ha il compito di interporsi tra l'LPA e il server SM-DP+ e provare a decriptare il traffico TLS mediante un attacco di tipo Man In The Middle (MITM). In tal modo, si è riusciti a ottenere una traccia (più precisamente tre tracce, ma ne è stata presa in considerazione una in particolare) in cui sono state superate le protezioni di TLS e che, quindi, contiene uno scambio di pacchetti dei quali è possibile accedere al payload, ovvero al contenuto dei messaggi dati dal protocollo RSP.\\
Si potrebbe pensare che, in questo caso specifico, il traffico sull'interfaccia ES9+ (tra SM-DP+ e LPA) sia verosimilmente protetto da TLS v1.2, sul quale dovrebbe risultare più facile performare un attacco MITM mediante un proxy relay rispetto alla versione 1.3 di TLS che invece è più sicura e risolve le vulnerabilità più evidenti che caratterizzano la versione 1.2.\\
Il file PCAP contenente la traccia così ottenuta è stato analizzato tramite Wireshark, che è un analizzatore di rete, e consente sia di mostrare tutti i dettagli relativi a uno scambio di pacchetti già salvato, sia di catturare a sua volta una traccia, almeno nel caso in cui il traffico di interesse coinvolga il sistema su cui il tool è installato.

\section{Analisi dei messaggi catturati}
Wireshark risulta fondamentale per individuare il formato esatto di ciascun pacchetto catturato all'interno della traccia. Innanzitutto è necessario stabilire quali sono i messaggi presenti all'interno del file PCAP per stabilire quale corrispondenza ci sia tra la comunicazione di riferimento tra l'LPA del dispositivo mobile e il server SM-DP+ di Very Mobile. In figura \ref{fig:all-msgs-pcap} è rappresentata la schermata di Wireshark che mostra la successione dei messaggi che sono stati scambiati durante la cattura.\\
\begin{figure}
\includegraphics[width=\linewidth]{all-msgs-pcap.png}
\caption{Lista di pacchetti catturati.}
\label{fig:all-msgs-pcap}
\end{figure}
Com'è possibile vedere, in mezzo a dei normalissimi pacchetti TCP, spiccano alcuni pacchetti contrassegnati dal protocollo HTTP/JSON, che sono relativi proprio ai messaggi che compongono la comunicazione tra l'LPA e il server SM-DP+. In particolare, ve ne sono tre inviati al server da parte del client, mentre gli altri tre non sono altro che le rispettive risposte del server. I tre messaggi inviati dal client sono elencati qui di seguito:
\begin{enumerate}
\item \textbf{initiateAuthentication}, che corrisponde al primo messaggio inviato al server SM-DP+ da parte dell'LPA durante la procedura di Common Mutual Authentication (sezione \ref{sec:mutual-auth}).
\item \textbf{authenticateClient}, che corrisponde al secondo messaggio inviato al server SM-DP+ da parte dell'LPA durante la procedura di Common Mutual Authentication (sezione \ref{sec:mutual-auth}).
\item \textbf{getBoundProfilePackage}, che corrisponde al primo messaggio inviato al server SM-DP+ da parte dell'LPA durante la sotto-procedura di Download Confirmation (sezione \ref{sec:down-confirm}).
\end{enumerate}
È chiaro che le risposte del server all'interno del file PCAP, nell'ordine, siano relative ai seguenti altri messaggi:
\begin{enumerate}
\item \textbf{initiateAuthenticationResponse}: corrisponde al messaggio inviato all'LPA da parte del server SM-DP+ allo step (9) della procedura di Common Mutual Authentication (sezione \ref{sec:mutual-auth}).
\item \textbf{authenticateClientResponse}, che corrisponde all'unico messaggio inviato all'LPA da parte del server SM-DP+ allo step (6) della procedura di download e installazione dei profili (sezione \ref{sec:down-install}).
\item \textbf{getBoundProfilePackageResponse}, che corrisponde al messaggio inviato all'LPA da parte del server SM-DP+ allo step (11) della sotto-procedura di Download Confirmation (sezione \ref{sec:down-confirm}).
\end{enumerate}
In realtà, le altre due tracce che sono state prese, oltre a questi sei pacchetti HTTP/JSON, riportano ulteriori messaggi contrassegnati dal medesimo protocollo, e sembrano corrispondere tutti a un \textbf{handleNotification} inviato al server SM-DP+ da parte dell'LPA in modo periodico: tale handleNotification è relativo all'ultimo messaggio ricevuto dal server durante la sotto-procedura di Download Confirmation (sezione \ref{sec:down-confirm}).\\
In definitiva, è possibile apprezzare un matching perfetto tra i pacchetti catturati nelle tracce PCAP e i messaggi previsti dalla documentazione di GSMA \cite{GSMA-docs-new} nel momento in cui si considerano in sequenza la procedura di Common Mutual Authentication e la procedura di download e installazione dei profili (inclusa la sotto-procedura di Download Confirmation).\\
Per comprendere appieno la struttura e il formato dei messaggi scambiati tra LPA e server SM-DP+, è certamente sufficiente effettuare un'analisi dettagliata del primo messaggio inviato dal client durante la procedura di Common Mutual Authentication (initiateAuthentication) e della relativa risposta del server (initiateAuthenticationResponse). Tuttavia, nella presente trattazione si intende effettuare un'analisi completa di tutti i pacchetti scambiati dalle controparti con lo scopo di acquisire più informazioni possibili sui messaggi scambiati ed, eventualmente, sul server SM-DP+ di Very Mobile e/o sul profilo installato nell'eUICC del dispositivo mobile.

\subsection{Analisi di initiateAuthentication}
Da Wireshark è possibile osservare il contenuto del messaggio initiateAuthentication all'interno della comunicazione reale tra LPA del dispositivo mobile e server SM-DP+ di Very Mobile. La figura \ref{fig:msg1-stream-pcap} ne mostra un estratto.\\
\begin{figure}
\includegraphics[width=\linewidth]{msg1-stream-pcap.png}
\caption{Messaggio di initiateAuthentication visualizzato su Wireshark.}
\label{fig:msg1-stream-pcap}
\end{figure}
L'header HTTP riporta alcune informazioni interessanti:
\begin{itemize}
\item il metodo HTTP utilizzato dall'LPA è POST;
\item la versione HTTP utilizzata è la 1.1;
\item il protocollo seguito per l'interazione tra LPA e server è la versione 2.2.0 di GSMA-RSP;
\item l'hostname del server SM-DP+ di Very Mobile è sys.prod.ondemandconnectivity.com;
\item l'LPA in questione accetta anche messaggi di risposta il cui payload è stato compresso tramite gzip.
\end{itemize}
Per quanto invece concerne il payload del messaggio, è possibile osservare che ha una struttura JSON con tre coppie chiave-valore. In particolare, i primi due campi, che sono euiccChallenge ed euiccInfo1, assumono dei valori rappresentati nella codifica \textbf{base64}, mentre il terzo e ultimo campo, ovvero smdpAddress, riporta in chiaro l'hostname del server SM-DP+ contattato.

\subsubsection{Codifica base64}
È un sistema di codifica che permette di convertire una sequenza di byte in una stringa in formato ASCII, dove i possibili caratteri diversi che compongono la stringa sono 64 \cite{RFC-4648}. Lo scopo della codifica è principalmente quello di rendere stampabile qualunque tipo di informazione.\\
L'idea alla base della codifica è molto semplice: si prende la sequenza di byte di partenza, la si suddivide in gruppi di 6 bit, e ciascun gruppo di 6 bit viene tradotto nel corrispondente carattere ASCII secondo la convenzione riportata nella figura \ref{fig:base64} tratta da \cite{RFC-4648}.\\
\begin{figure}
\includegraphics[width=\linewidth]{base64.png}
\caption{Tabella di conversione della codifica base64.}
\label{fig:base64}
\end{figure}
La tabella di conversione della codifica base64 riporta anche un sessantacinquesimo carattere ('='), che svolge la funzione di padding. Infatti, è possibile assumere che ogni sequenza di byte data in input all'algoritmo di codifica base64 abbia una lunghezza multipla di 8 bit, ma non è detto che abbia una lunghezza multipla anche di 6 bit (i.e. di 24 bit). Per questo motivo, si hanno tre casistiche \cite{RFC-4648}:
\begin{enumerate}
\item La sequenza di byte di input ha una lunghezza multipla di 24 bit: in tal caso, non è necessario avere particolari accortezze e, in particolare, non è previsto alcun padding.
\item La sequenza di byte di input ha una lunghezza tale per cui alla fine avanzano 8 bit: in tal caso, vengono aggiunti 4 bit di padding pari a 0, vengono convertiti i 12 bit così ottenuti nei corrispettivi 2 caratteri ASCII e, in coda alla stringa di output, vengono aggiunti 2 caratteri di uguale ('==').
\item La sequenza di byte di input ha una lunghezza tale per cui alla fine avanzano 16 bit: in tal caso, vengono aggiunti 2 bit di padding pari a 0, vengono convertiti i 18 bit così ottenuti nei corrispettivi 3 caratteri ASCII e, in coda alla stringa di output, viene aggiunto un carattere di uguale ('=').
\end{enumerate}
Sulla base dell'algoritmo di conversione appena esposto, è possibile concludere che il campo euiccChallenge del messaggio è uguale al valore intero 232645733703218863597835671721642336624, mentre il campo euiccInfo1 risulta pari alla sequenza di byte mostrata in figura \ref{fig:asn1-euiccInfo1}.\\
\begin{figure}
\includegraphics[width=\linewidth]{asn1-euiccInfo1.png}
\caption{Sequenza di byte che descrive il campo euiccInfo1 di initiateAuthentication.}
\label{fig:asn1-euiccInfo1}
\end{figure}
Le specifiche di GSMA \cite{GSMA-docs-new} suggeriscono che i tipi di dato complessi come euiccInfo1 in RSP vengono rappresentati secondo lo standard \textbf{ASN.1} sfruttando la codifica \textbf{DER}.

\subsubsection{Codifica DER}
DER (Distinguished Encoding Rules) è una variante restrittiva di \textbf{BER} (Basic Encoding Rules) che, di fatto, elimina alcune flessibilità offerte da BER \cite{DER}.\\
BER a sua volta rappresenta un insieme di regole per la conversione di dati eterogenei in flussi di byte. Per la codifica, fa uso del formato \textbf{TLV} (Tag-Length-Value) e, per questa ragione, è particolarmente legato allo standard ASN.1 \cite{BER}. Di conseguenza, per comprendere al meglio il significato dei byte che compongono euiccInfo1, può risultare utile introdurre una panoramica generale sul funzionamento dello standard ASN.1.

\subsubsection{Standard ASN.1}
ASN.1 (Abstract Syntax Notation One) è uno standard di codifica che rappresenta le informazioni sfruttando il formato TLV, proprio come previsto dalle specifiche della codifica BER \cite{RFC-6025}\cite{ASN1}. Ciò significa che ciascuna unità di informazione è suddivisa in tre parti:
\begin{itemize}
\item \textbf{Tag}: detto anche \textbf{type}, è composto tipicamente da un byte e indica il tipo di dato che si sta rappresentando. Il tipo può essere primitivo (e.g. Boolean, Integer, UTF8String \cite{ASN1-types}), composto (e.g. Sequence, Set \cite{ASN1-types}), un metadato (e.g. Object Identifier \cite{ASN1-types}) oppure un tipo di dato custom; in quest'ultimo caso, spesso e volentieri il tag è composto da due byte anziché da uno.
\item \textbf{Length}: è un campo che può essere composto da uno o più byte. Più precisamente, si hanno tre casi \cite{ASN1}:
\begin{itemize}[itemsep=0pt]
\item Campo length composto da un solo byte compreso tra 0x00 e 0x7F (i.e. tra 0 e 127): in tal caso l'unico byte rappresenta la lunghezza effettiva del campo value.
\item Campo length composto da più di un byte: in tal caso il primo byte deve essere compreso tra 0x81 e 0xFF (i.e. tra 129 e 255) ed è tale per cui il suo valore diminuito di 0x80 (i.e. 128) rappresenta la lunghezza della restante porzione del campo length. La restante porzione del campo length, invece, indica la lunghezza effettiva del campo value. Questa convenzione viene applicata tutte e sole le volte in cui il campo value è più lungo di 127 byte. Per fare un esempio, per indicare una lunghezza di value pari a 257 byte, il campo length deve essere uguale a 0x820101: il primo byte (0x82) indica che, oltre a lui stesso, si hanno ulteriori due byte (i.e. 0x82-0x80) a comporre il campo length; gli ultimi due byte (0x0101), invece, stanno a indicare esattamente la lunghezza del campo value, che è proprio pari a 257.
\item Campo length il cui primo byte è pari a 0x80: in tal caso, si ha una length "indefinita", per cui è verosimilmente sintomo di una formattazione errata.
\end{itemize}
\item \textbf{Value}: è il campo che contiene il payload vero e proprio dell'informazione che si sta rappresentando. Nel caso particolare in cui il relativo tipo di dato è composto, tale campo è costituito a sua volta da altri elementi TLV, creando così un'annidazione che può essere ricorsiva a piacere \cite{ASN1}.
\end{itemize}
Comunque sia, avere una conoscenza accurata dello standard ASN.1 non è sufficiente per interpretare in modo agevole le sequenze di byte come quella in figura \ref{fig:asn1-euiccInfo1} che è relativa a euiccInfo1. Infatti, a primo impatto si intuisce solo che euiccInfo1 è composto a sua volta da tre elementi in formato TLV (che iniziano rispettivamente con i byte 0x8203, 0xA92C, 0xAA2C), ma il loro significato non è chiaro. Per questo motivo, si ricorre alle definizioni ASN.1 in RSP, che è possibile trovare in \cite{RSP-definitions} (vedere repository \textit{sim-card/gsma-esim-iot}). In figura \ref{fig:def-euiccInfo1} è ritratta la definizione precisa della struttura dati euiccInfo1 tratta da \cite{RSP-definitions}.\\
\begin{figure}
\includegraphics[width=\linewidth]{def-euiccInfo1.png}
\caption{Definizione del campo euiccInfo1 di initiateAuthentication.}
\label{fig:def-euiccInfo1}
\end{figure}
Come si può evincere, euiccInfo1 è una struttura dati composta da alcune informazioni che caratterizzano l'eUICC del dispositivo mobile coinvolto nella comunicazione. Tali informazioni sono schematizzate di seguito.
\begin{itemize}
\item lo Specification Version Number (SVN), che sarebbe la versione di GSMA SGP.22 supportata dall'eUICC;
\item una lista (Sequence) di identificatori delle chiavi pubbliche del CI che possono essere utilizzate dall'eUICC per autenticare il server;
\item una lista (Sequence) di identificatori delle chiavi pubbliche del CI che possono essere utilizzate dall'eUICC per firmare i propri dati.
\end{itemize}
Si noti che questa definizione di euiccInfo1 combacia perfettamente con le specifiche della GSMA \cite{GSMA-docs-new} relative alla procedura di Common Mutual Authentication \ref{sec:mutual-auth}.\\
Per determinare nel modo più agevole possibile i valori assunti dai tre campi di euiccInfo1 sopra elencati, si può far riferimento al sito https://lapo.it/asn1js/ (\textit{ASN.1 JavaScript decoder}) il quale, data una stringa base64, restituisce in chiaro le informazioni codificate con lo standard ASN.1. In figura \ref{fig:decode-euiccInfo1} è riportato l'output del sito \textit{ASN.1 JavaScript decoder} nel momento in cui gli viene dato in pasto la stringa base64 associata a euiccInfo1 in figura \ref{fig:msg1-stream-pcap}.\\
\begin{figure}
\includegraphics[width=\linewidth]{decode-euiccInfo1.png}
\caption{Decodifica del campo euiccInfo1 di initiateAuthentication.}
\label{fig:decode-euiccInfo1}
\end{figure}
In definitiva:
\begin{itemize}
\item SVN è pari alla sequenza di byte 0x020202, che verosimilmente sta a indicare la versione 2.2.2;
\item euiccCiPKIdListForVerification ed euiccCiPKIdListForSigning sono due liste identiche ciascuna delle quali contiene due identificatori di chiavi pubbliche del CI.
\end{itemize}

\subsection{Analisi di initiateAuthenticationResponse}
La figura \ref{fig:msg2-stream-pcap} mostra l'intero contenuto del primo messaggio di risposta del server SM-DP+.\\
\begin{figure}
\includegraphics[width=\linewidth]{msg2-stream-pcap.png}
\caption{Messaggio di initiateAuthenticationResponse visualizzato su Wireshark.}
\label{fig:msg2-stream-pcap}
\end{figure}
Nell'header HTTP si ritrovano alcuni punti in comune col messaggio inviato dall'LPA: ad esempio, anche qui la versione HTTP utilizzata è la 1.1 e il protocollo usato per l'interazione tra LPA e server è la versione 2.2.0 di GSMA-RSP.\\

Il payload del messaggio ha una struttura JSON del tutto analoga a quella di initiateAuthentication. In particolare, si hanno i seguenti campi:
\begin{itemize}
\item \textbf{header}: è un campo in chiaro che indica l'esito del processamento del messaggio dell'LPA al livello del protocollo RSP. La presenza di \textit{\{"status":"Executed-Success"\}} indica che tutti i check sulle informazioni ricevute dal client sono andati a buon fine e la comunicazione può proseguire secondo le specifiche GSMA;
\item \textbf{transactionId}: è un campo in chiaro che indica l'ID della transazione;
\item \textbf{serverSigned1}: è un campo codificato in base64;
\item \textbf{serverSignature1}: è un altro campo codificato in base64 che, come enunciato nella sezione \ref{sec:mutual-auth}, rappresenta la signature di serverSigned1;
\item \textbf{euiccCiPKIdToBeUsed}: è il terzo campo codificato in base64;
\item \textbf{serverCertificate}: è l'ultimo campo codificato in base64. La sua lunghezza è giustificata dal fatto che, con riferimento alla sezione \ref{sec:mutual-auth}, rappresenta il certificato CERT.DPauth.SIG, che è verosimilmente una struttura dati molto complessa.
\end{itemize}
Per comprendere appieno le informazioni celate nei campi di initiateAuthenticationResponse codificati in base64, si fa nuovamente riferimento a \cite{RSP-definitions} e al sito \textit{ASN.1 JavaScript decoder}.\\

La struttura dati serverSigned1 è molto semplice: dalle figure \ref{fig:def-serverSigned1}, \ref{fig:decode-serverSigned1} \cite{RSP-definitions}, si capisce che è composta dall'ID di transazione, dalla challenge (euiccChallenge) che il server aveva ricevuto in initiateAuthentication, dall'hostname del server e da una nuova challenge generata dal server (serverChallenge).\\
\begin{figure}
\includegraphics[width=\linewidth]{def-serverSigned1.png}
\caption{Definizione del campo serverSigned1 di initiateAuthenticationResponse.}
\label{fig:def-serverSigned1}
\end{figure}
\begin{figure}
\includegraphics[width=\linewidth]{decode-serverSigned1.png}
\caption{Decodifica del campo serverSigned1 di initiateAuthenticationResponse.}
\label{fig:decode-serverSigned1}
\end{figure}

Il campo euiccCiPKIdToBeUsed cela anch'esso una struttura TLV dello standard ASN.1, com'è possibile vedere dalle figure \ref{fig:asn1-euiccCiPKIdToBeUsed}, \ref{fig:decode-euiccCiPKIdToBeUsed}. Inoltre, il relativo campo value assume un valore che è comparso sia in euiccCiPKIdListForVerification, sia in euiccCiPKIdListForSigning, il che vuol dire che il server SM-DP+ accetta l'utilizzo di una delle due chiavi pubbliche di CA (i.e. una delle due certificate chain) proposte dal client nel messaggio initiateAuthentication.\\
\begin{figure}
\includegraphics[width=\linewidth]{asn1-euiccCiPKIdToBeUsed.png}
\caption{Sequenza di byte che descrive il campo euiccCiPKIdToBeUsed di initiateAuthenticationResponse.}
\label{fig:asn1-euiccCiPKIdToBeUsed}
\end{figure}
\begin{figure}
\includegraphics[width=\linewidth]{decode-euiccCiPKIdToBeUsed.png}
\caption{Decodifica del campo euiccCiPKIdToBeUsed di initiateAuthenticationResponse.}
\label{fig:decode-euiccCiPKIdToBeUsed}
\end{figure}

Per quanto invece riguarda il campo serverCertificate di initiateAuthenticationResponse, il sito \textit{ASN.1 JavaScript decoder} aiuta a comprendere molti dettagli sul certificato CERT.DPauth.SIG e sulla relativa certificate chain utilizzata. Le figure \ref{fig:decode-serverCertificate1}, \ref{fig:decode-serverCertificate2} mostrano la decodifica di tale certificato. Inizialmente vi sono gli attributi fondamentali, come:
\begin{itemize}
\item \textbf{signature AlgorithmIdentifier}: indica con quale algoritmo vengono effettuate le signature ed è pari a \textit{ecdsaWithSHA256} (ANSI X.962 ECDSA algorithm with SHA256);
\item \textbf{issuer Name}: è un campo relativo alla CA che ha rilasciato il presente certificato ed è a sua volta articolato in due sotto-campi, che sono l'organizationName (pari a "GSM Association") e il commonName (pari a "GSM Association - RSP2 Root CI1");
\item \textbf{validity Validity}: dichiara che il certificato è valido dal 30 marzo 2023 al 29 marzo 2026;
\item \textbf{subject Name}: è un campo che fornisce dettagli informativi sul presente certificato ed è a sua volta composto in diversi sotto-campi, tra cui countryName (pari a "FR", i.e. Francia), localityName (pari a "Tours"), organizationName (pari a "Thales DIS France SA"), organizationalUnitName (pari a "ODC") e commonName (pari a "SMDP Plus Tours Platform").
\item \textbf{subjectPublicKeyInfo SubjectPublicKeyInfo}: fornisce maggiori informazioni sulla chiave pubblica associata al certificato, tra cui la curva ellittica utilizzata per generare le signature, che è identificata con \textit{prime256v1}.
\end{itemize}
Più in basso nel certificato, si trovano le cosiddette estensioni, ovvero informazioni aggiuntive opzionali che possono essere aggiunte a piacere all'interno del certificato. Tra queste si hanno:
\begin{itemize}
\item \textbf{subjectKeyIdentifier}: è l'identificatore della chiave pubblica associata al presente certificato;
\item \textbf{authorityKeyIdentifier}: è l'identificatore della chiave pubblica dell'issuer del certificato e, per coerenza, deve corrispondere o contenere il valore associato al campo euiccCiPKIdToBeUsed del messaggio initiateAuthenticationResponse;
\item \textbf{keyUsage}: rappresenta i possibili utilizzi delle chiavi associate al certificato;
\item \textbf{subjectAltName}: rappresenta il nome alternativo dell'entità che possiede il certificato;
\item altre informazioni che non verranno esaminate a fondo in questa trattazione, come certificatePolicies e cRLDistributionPoints.
\end{itemize}
Infine, in fondo al certificato, si hanno ulteriori informazioni riguardanti la signature e l'algoritmo di signature.
\begin{figure}
\includegraphics[width=\linewidth]{decode-serverCertificate1.png}
\caption{Decodifica del campo serverCertificate di initiateAuthenticationResponse (parte 1/2).}
\label{fig:decode-serverCertificate1}
\end{figure}
\begin{figure}
\includegraphics[width=\linewidth]{decode-serverCertificate2.png}
\caption{Decodifica del campo serverCertificate di initiateAuthenticationResponse (parte 2/2).}
\label{fig:decode-serverCertificate2}
\end{figure}

\subsection{Analisi di authenticateClient}
La figura \ref{fig:msg3-stream-pcap} mostra il contenuto del secondo messaggio inviato dal client (authenticateClient).\\
\begin{figure}
\includegraphics[width=\linewidth]{msg3-stream-pcap.png}
\caption{Messaggio di authenticateClient visualizzato su Wireshark.}
\label{fig:msg3-stream-pcap}
\end{figure}
Il JSON del payload è composto da due campi:
\begin{itemize}
\item \textbf{authenticateServerResponse}: è una struttura dati che verrà approfondita tra poco;
\item \textbf{transactionId}: corrisponde all'ID di transazione appena ricevuto nel messaggio initiateAuthenticationResponse.
\end{itemize}
Per quanto riguarda authenticateServerResponse, è una struttura composta da più elementi ASN.1 di tipo Sequence anche annidati tra loro, ed è descritta all'interno delle figure \ref{fig:def-authenticateServerResponse1}, \ref{fig:def-authenticateServerResponse2}, \ref{fig:def-authenticateServerResponse3} \cite{RSP-definitions}.\\
\begin{figure}
\includegraphics[width=\linewidth]{def-authenticateServerResponse1.png}
\caption{Definizione del campo authenticateServerResponse di authenticateClient.}
\label{fig:def-authenticateServerResponse1}
\end{figure}
\begin{figure}
\includegraphics[width=\linewidth]{def-authenticateServerResponse2.png}
\caption{Definizione del sotto-campo euiccInfo2 di EuiccSigned1.}
\label{fig:def-authenticateServerResponse2}
\end{figure}
\begin{figure}
\includegraphics[width=\linewidth]{def-authenticateServerResponse3.png}
\caption{Definizione del sotto-campo ctxParams1 di EuiccSigned1.}
\label{fig:def-authenticateServerResponse3}
\end{figure}
Com'è possibile osservare, authenticateServerResponse si articola in due casi: authenticateResponseOk nel caso in cui l'autenticazione del server SM-DP+ sia andata a buon fine, e authenticateResponseError in caso contrario. La struttura authenticateResponseError è composta semplicemente dall'ID di transazione e da un codice di errore \cite{RSP-definitions}, mentre la struttura authenticateResponseOk è molto più articolata.\\
Anzitutto, authenticateResponseOk è composto da quattro campi: euiccSigned1 che è a sua volta un'altra struttura dati complessa, euiccSignature1 che, come ormai suggerisce il nome, rappresenta la firma digitale di euiccSigned1, il certificato dell'eUICC (CERT.EUICC.SIG) e il certificato dell'EUM (CERT.EUM.SIG). La struttura euiccSigned1 si articola in:
\begin{itemize}
\item \textbf{transactionId}: è l'ID della transazione;
\item \textbf{serverAddress}: è l'hostname del server SM-DP+;
\item \textbf{serverChallenge}: è la challenge generata precedentemente dal server e comunicata al client mediante il messaggio initiateAuthenticationResponse;
\item \textbf{euiccInfo2}: è un'ulteriore struttura dati che comprende tutte le informazioni sull'eUICC che si sta per autenticare e su cui avverrà l'installazione del nuovo profilo. È un sovrainsieme del campo euiccInfo1 di initiateAuthentication;
\item \textbf{ctxParams1}: come suggerito dal nome, rappresenta i parametri di contesto e, in particolar modo, le informazioni del dispositivo e il Matching ID. Quest'ultimo è tipicamente un token associato all'Activation Code (codice di attivazione) \cite{RSP-definitions}, il quale viene negoziato da server SM-DP+ e operatore durante la fase di \textit{profile ordering} e viene rilasciato dall'operatore mediante un'apposita email durante la fase di \textit{download initialization} (sezione \ref{sec:step-RSP}). Un estratto dell'email di Very Mobile relativa al profilo che si sta per installare sull'eUICC del dispositivo mobile di riferimento è riportato in figura \ref{fig:email}.
\end{itemize}
\begin{figure}
\includegraphics[width=\linewidth]{email.png}
\caption{Estratto dell'email di Very Mobile.}
\label{fig:email}
\end{figure}
La figura \ref{fig:decode-euiccInfo2} mostra come il sito \textit{ASN.1 JavaScript decoder} decodifica il campo euiccSigned1 a partire dalla stringa base64 relativa all'ASN.1 del campo authenticateServerResponse del messaggio authenticateClient. A partire dalla sestultima riga dell'immagine si ha la traduzione della struttura ctxParamas1: qui si nota la corrispondenza tra il sotto-campo matchingId e l'Activation Code presente nell'email di cui sopra.
\begin{figure}
\includegraphics[width=\linewidth]{decode-euiccInfo2.png}
\caption{Decodifica del campo euiccInfo2 di authenticateServerResponse.}
\label{fig:decode-euiccInfo2}
\end{figure}

\subsection{Analisi di authenticateClientResponse}
La figura \ref{fig:msg4-stream-pcap} mostra il contenuto della risposta del server al secondo messaggio inviato dal client (authenticateClientResponse).\\
\begin{figure}
\includegraphics[width=\linewidth]{msg4-stream-pcap.png}
\caption{Messaggio di authenticateClientResponse visualizzato su Wireshark.}
\label{fig:msg4-stream-pcap}
\end{figure}
Il JSON del payload è composto dai campi elencati qui di seguito:
\begin{itemize}
\item \textbf{header}: contiene di nuovo \textit{\{"status":"Executed-Success"\}}, il che vuol dire che anche in questa fase tutti i check effettuati dal server SM-DP+ hanno avuto successo;
\item \textbf{transactionId}: è l'ID di transazione e assume sempre il medesimo valore;
\item \textbf{profileMetadata}: è un campo codificato in base64 che, come suggerisce il nome, dovrebbe comprendere i metadati del profilo di cui si sta per effettuare il download;
\item \textbf{smdpSigned2}: è un campo codificato in base 64 che, come suggerisce il nome, dovrebbe essere strutturalmente simile a serverSigned1 e a euiccSigned1;
\item \textbf{smdpSignature2}: è un campo codificato in base64 relativo alla signature di smdpSigned2;
\item \textbf{smdpCertificate}: è un ultimo campo codificato in base64 che, con riferimento alla sezione \ref{sec:down-install}, rappresenta il certificato CERT.DPpb.SIG, che è potenzialmente diverso dal certificato CERT.DPauth.SIG inviato all'LPA precedentemente e viene utilizzato per firmare il profilo che si sta per scaricare.
\end{itemize}
I campi che possono essere approfonditi con l'aiuto di \cite{RSP-definitions} e del sito \textit{ASN.1 JavaScript decoder} sono profileMetadata e smdpSigned2.\\

La struttura dati profileMetadata, come riportato in figura \ref{fig:def-profileMetadata}, comprende informazioni sul profilo target come l'ICCID (che, come spiegato nella sezione \ref{sec:device-change}, è l'identificativo del profilo), il nome dell'operatore, il nome del profilo, la classe del profilo ed eventuali Profile Policy Rules.\\

La struttura dati smdpSigned2, come mostrato in figura \ref{fig:def-smdpSigned2}, contiene l'ID di transazione e un booleano che indica se è richiesto il Confirmation Code immesso dall'utente per completare il download e l'installazione del profilo. La necessità o meno del Confirmation Code e l'eventuale relativo valore vengono negoziati da server SM-DP+ e operatore durante la fase di \textit{profile ordering} (sezione \ref{sec:step-RSP}). Nel caso specifico del profilo scaricato con l'operatore Very Mobile, il Confirmation Code non è richiesto, come testimoniato dalla figura \ref{fig:decode-smdpSigned2} tratta dal sito \textit{ASN.1 JavaScript decoder}.\\
\begin{figure}
\includegraphics[width=\linewidth]{def-profileMetadata.png}
\caption{Definizione del campo profileMetadata di authenticateClientResponse.}
\label{fig:def-profileMetadata}
\end{figure}
\begin{figure}
\includegraphics[width=\linewidth]{def-smdpSigned2.png}
\caption{Definizione del campo smdpSigned2 di authenticateClientResponse.}
\label{fig:def-smdpSigned2}
\end{figure}
\begin{figure}
\includegraphics[width=\linewidth]{decode-smdpSigned2.png}
\caption{Decodifica del campo smdpSigned2 di authenticateClientResponse.}
\label{fig:decode-smdpSigned2}
\end{figure}

\subsection{Analisi di getBoundProfilePackage}
La figura \ref{fig:msg5-stream-pcap} mostra il contenuto del terzo messaggio inviato dal client (getBoundProfilePackage).\\
\begin{figure}
\includegraphics[width=\linewidth]{msg5-stream-pcap.png}
\caption{Messaggio di getBoundProfilePackage visualizzato su Wireshark.}
\label{fig:msg5-stream-pcap}
\end{figure}
Il payload è composto da due campi:
\begin{itemize}
\item \textbf{prepareDownloadResponse}: è una struttura dati in base64 che verrà descritta a breve;
\item \textbf{transactionId}: è il solito ID di transazione.
\end{itemize}
La definizione di prepareDownloadResponse è riportata in figura \ref{fig:def-prepareDownloadResponse} \cite{RSP-definitions}.\\
\begin{figure}
\includegraphics[width=\linewidth]{def-prepareDownloadResponse.png}
\caption{Definizione del campo prepareDownloadResponse di getBoundProfilePackage.}
\label{fig:def-prepareDownloadResponse}
\end{figure}
Anche prepareDownloadResponse, come authenticateServerResponse nel messaggio authenticateClient, si articola in due casi: prepareDownloadResponseOk nel caso in cui tutti i controlli effettuati dal client alla ricezione di authenticateClientResponse abbiano avuto successo, e prepareDownloadResponseError in caso contrario. La struttura dati prepareDownloadResponseError, in modo perfettamente analogo ad authenticateResponseError, è composta dall'ID di transazione e da un codice di errore.\\
D'altra parte, prepareDownloadResponseOk è composto dalla struttura dati euiccSigned2 e dalla relativa digital signature euiccSignature2, dove euiccSigned2 è composto a sua volta da:
\begin{itemize}
\item \textbf{transactionId}: è l'ID di transazione;
\item \textbf{euiccOtpk}: è la chiave pubblica otPK.EUICC.KA (sezione \ref{sec:keys});
\item \textbf{hashCc}: è un campo opzionale che è presente se il Confirmation Code è richiesto e corrisponde al valore SHA256(SHA256(Confirmation Code) | transactionId).
\end{itemize}

\subsection{Analisi di getBoundProfilePackageResponse}
La figura \ref{fig:msg6-stream-pcap} mostra il contenuto della risposta del server al terzo messaggio inviato dal client (getBoundProfilePackageResponse).\\
\begin{figure}
\includegraphics[width=\linewidth]{msg6-stream-pcap.png}
\caption{Messaggio di getBoundProfilePackageResponse visualizzato su Wireshark.}
\label{fig:msg6-stream-pcap}
\end{figure}
Il payload è composto dai seguenti tre campi:
\begin{itemize}
\item \textbf{header}: contiene ancora una volta \textit{\{"status":"Executed-Success"\}};
\item \textbf{transactionId}: è l'ID di transazione;
\item \textbf{boundProfilePackage}: è un campo codificato in base64 ed è molto grande, tant'è vero che non rientra per intero nella figura \ref{fig:msg6-stream-pcap}.
\end{itemize}
Com'è possibile osservare in figura \ref{fig:def-boundProfilePackage} \cite{RSP-definitions}, il campo boundProfilePackage è composto da strutture di tipo TLV, ovvero sequenze di byte che a loro volta possono essere interpretate come elementi ASN.1 coi loro tag, le loro length e i loro value. I lettori più attenti, dalla figura \ref{fig:decode-boundProfilePackage}, si accorgeranno che il campo sequenceOf88 di boundProfilePackage, che in questo caso specifico corrisponde al campo da 109 byte, contiene una sequenza di byte che corrisponde all'intero TLV di profileMetadata.\\
\begin{figure}
\includegraphics[width=\linewidth]{def-boundProfilePackage.png}
\caption{Definizione del campo boundProfilePackage di getBoundProfilePackageResponse.}
\label{fig:def-boundProfilePackage}
\end{figure}
\begin{figure}
\includegraphics[width=\linewidth]{decode-boundProfilePackage.png}
\caption{Decodifica del campo boundProfilePackage di getBoundProfilePackageResponse.}
\label{fig:decode-boundProfilePackage}
\end{figure}

\subsection{Analisi di handleNotification}
Nelle due tracce diverse da quella di riferimento sono stati catturati anche dei messaggi di handleNotification inviati al server SM-DP+ da parte dell'LPA. Può tornare utile analizzarne un campione, come quello riportato in figura \ref{fig:msg7-stream-pcap}.\\
\begin{figure}
\includegraphics[width=\linewidth]{msg7-stream-pcap.png}
\caption{Messaggio di handleNotification visualizzato su Wireshark.}
\label{fig:msg7-stream-pcap}
\end{figure}
Si ha un solo campo JSON, pendingNotification, la cui definizione è distribuita nelle figure \ref{fig:def-pendingNotification1}, \ref{fig:def-pendingNotification2}, \ref{fig:def-pendingNotification3} \cite{RSP-definitions}.\\
\begin{figure}
\includegraphics[width=\linewidth]{def-pendingNotification1.png}
\caption{Definizione del campo pendingNotification di handleNotification (parte 1/3).}
\label{fig:def-pendingNotification1}
\end{figure}
\begin{figure}
\includegraphics[width=\linewidth]{def-pendingNotification2.png}
\caption{Definizione del campo pendingNotification di handleNotification (parte 2/3).}
\label{fig:def-pendingNotification2}
\end{figure}
\begin{figure}
\includegraphics[width=\linewidth]{def-pendingNotification3.png}
\caption{Definizione del campo pendingNotification di handleNotification (parte 3/3).}
\label{fig:def-pendingNotification3}
\end{figure}
Si hanno due casi possibili: pendingNotification pari alla struttura dati profileInstallationResult e pendingNotification pari alla struttura dati otherSignedNotification.
\begin{itemize}
\item Nel primo caso, si ha una struttura profileInstallationResultData composta da ID di transazione, alcuni metadati di notificazione, OID (i.e. l'identificatore) del server SM-DP+ ed esito finale dell'installazione del profilo all'interno dell'eUICC (che può essere un successResult o un errorResult). Oltre a profileInstallationResultData, il campo pendingNotification è composto anche da euiccSignPIR, che risulta essere la signature di profileInstallationResultData.
\item Nel secondo caso, invece, si hanno alcuni metadati di notificazione, la relativa digital signature, il certificato dell'eUICC (CERT.EUICC.SIG) e il certificato dell'EUM (CERT.EUM.SIG). Intuitivamente, si ricade in questa casistica in contesti differenti da quello in cui l'LPA ha bisogno di comunicare l'esito dell'installazione del profilo (i.e. il successResult o l'errorResult).
\end{itemize}
Poiché, com'è stato già visto, la struttura dei certificati è piuttosto grande e poiché la stringa in base64 che codifica il campo pendingNotification del messaggio handleNotification, al contrario, è molto breve, è già possibile intuire che il messaggio catturato nella traccia ricada nel primo caso esposto precedentemente, dove pendingNotification è pari alla struttura dati profileInstallationResult che, verosimilmente, contiene un successResult. Tale intuizione trova conferma in figura \ref{fig:decode-pendingNotification} tratta dal sito \textit{ASN.1 JavaScript decoder}, dove chiaramente non compaiono riferimenti a certificati.
\begin{figure}
\includegraphics[width=\linewidth]{decode-pendingNotification.png}
\caption{Decodifica del campo pendingNotification di handleNotification.}
\label{fig:decode-pendingNotification}
\end{figure}

\chapter{Risultati ottenuti}
[TODO]

\chapter{Conclusione}
[TODO]

\bibliography{Bibliography}

\chapter*{Ringraziamenti}
[TODO]

\end{document}